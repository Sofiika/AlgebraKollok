\documentclass{article}

\usepackage{fancybox,fancyhdr}
\linespread{1.15}
\usepackage{amsfonts}
\usepackage{amsmath}
\usepackage[utf8]{inputenc}
\usepackage[russian]{babel}
\usepackage{hhline}
\usepackage{multirow}
\usepackage{setspace}
\usepackage[final]{graphicx}
\usepackage[usenames]{color}
\usepackage{colortbl}
\usepackage{latexsym}
\usepackage{wasysym}
\usepackage{MnSymbol}

\usepackage{graphicx}  % Для вставки рисунков
\graphicspath{{images/}{images2/}}  % папки с картинками
\setlength\fboxsep{3pt} % Отступ рамки \fbox{} от рисунка
\setlength\fboxrule{1pt} % Толщина линий рамки \fbox{}
\usepackage{wrapfig} % Обтекание рисунков и таблиц текстом

\usepackage[usenames,dvipsnames,svgnames,table]{xcolor}
\usepackage{lettrine} %%% To make a Drop Cap
\usepackage{yfonts} %% to make a fancy Gothic drop caps.

\textheight=24.3cm
\textwidth=19cm
\oddsidemargin=-1.2cm
\topmargin=-2cm

\begin{document}
	
	\pagestyle{fancy}
	\fancyhead{}
	\fancyhead[C]{\normalsize\color{gray}{Коллоквиум определения, 2019 г.}}
	\fancyhead[R]{\normalsize\color{gray}{vk.com/hse$\_$botai, vk.com/zdarovablin}}
	\fancyhead[L]{\normalsize\color{gray}{ПИ, линейная алгебра}}  
	\fancyfoot[C]{\normalsize\color{gray}{\thepage}}
	\renewcommand{\footrulewidth}{0.1 mm}
	\Large
	\centering
	
	\textbf{1-й модуль}
	
	\flushleft
	\small
	
	\textbf{1. Дать определение умножения матриц. Коммутативна ли эта операция? Ответ пояснить.} 
	
		
	{
		$\;$
		\setlength{\parindent}{0.4cm}
		\hangindent=0.4cm
	
		\textit{Произведением матриц} $A_{n\times p}$ и $B_{p\times k}$ называется матрица $C$ типа $n\times k$, где 
		$c_{ij}=\sum\limits_{l=1}^p a_{il}\cdot b_{lj}$. Умножение матриц, вообще говоря, не коммутативно, то есть $A\cdot B$, вообще говоря, $\ne B\cdot A$.
	
		$\;$
		
		\textbf{Пример:}
		$$
		A=\begin{pmatrix} 
		0 & 1\\ 
		0 & 0 
		\end{pmatrix}, \; B=\begin{pmatrix} 
		0 & 0\\ 
		0 & 1 
		\end{pmatrix}
		\qquad
		A\cdot B=\begin{pmatrix} 
		0 & 1\\ 
		0 & 0 
		\end{pmatrix}, \; B\cdot A=\begin{pmatrix} 
		0 & 0\\ 
		0 & 0 
		\end{pmatrix}$$
		$\;$
		\setlength{\parindent}{0cm}
		\hangindent=0cm
	}
	
	\textbf{2. Дать определение ступенчатого вида матрицы и канонического вида матрицы.} 
	
	{
		$\;$
		\setlength{\parindent}{0.4cm}
		\hangindent=0.4cm
		
		Матрица $M$ имеет \textit{ступенчатый вид}, если номера первых ненулевых элементов всех строк (такие элементы называют ведущими) возрастают, а нулевые строки стоят внизу матрицы. 
		
		$\;$
		
		Матрица $M$ имеет \textit{канонический} вид, если $M$ уже имеет ступенчатый вид, причем все ведущие элементы равны 1 и в любом столбце, содержащем ведущий элемент, выше и ниже него стоят 0.
		
		$\;$
		\setlength{\parindent}{0cm}
		\hangindent=0cm
	}
	
	\textbf{3. Перечислить элементарные преобразования строк.} 
	
	{
		$\;$
		\setlength{\parindent}{0.4cm}
		\hangindent=0.4cm
		
		Пусть $(i)$ -- $i$-тая строка матрицы $A$.
		
		Тогда элементарные преобразования:
		
		1$\left.\right)$ $(i)\rightarrow\lambda\cdot(i)$, $ \lambda\ne 0$ --
		умножили $i$-тую строку на число $\lambda$
		
		2$\left.\right)$ $(i)\leftrightarrow (j)$ -- поменяли местами $i$-тую и $j$-тую строки
		
		3$\left.\right)$ $(i)\rightarrow (i)+\lambda\cdot(k)$ -- $i$-тая строка заменяется на сумму $i$-той строки и $k$-той		
строки $\cdot$ число $\lambda$
		
		$\;$
		\setlength{\parindent}{0cm}
		\hangindent=0cm
	}

	\textbf{4. Сформулировать теорему о методе Гаусса (алгоритм приводить не нужно).}
	
	{
		$\;$
		\setlength{\parindent}{0.4cm}
		\hangindent=0.4cm
		
		Любую конечную матрицу $A$ можно привести элементарными преобразованиями к ступенчатому (каноническому) виду.
		
		$\;$
		\setlength{\parindent}{0cm}
		\hangindent=0cm
	}
	
	\textbf{5. Дать определения перестановки и подстановки.}
	
	{
		$\;$
		\setlength{\parindent}{0.4cm}
		\hangindent=0.4cm

		Всякое расположение чисел от 1 до $n$ в определенном порядке называют \textit{перестановкой}. 
		
		$\;$
		
		\textit{Подстановка} $\sigma$ {\scriptsize $\begin{pmatrix}
		1& \ldots& n\\
		\sigma(1)&\ldots&\sigma(n)
		\end{pmatrix}$} -- отображение множества $1, \ldots, n$ в себя. Это отображение должно быть биективным.
		
		$\;$
		\setlength{\parindent}{0cm}
		\hangindent=0cm
	}
	
	
	\textbf{6. Дать определение знака и четности подстановки.} 
	
	{
		$\;$
		\setlength{\parindent}{0.4cm}
		\hangindent=0.4cm
		
		Знак подстановки {\scriptsize $\begin{pmatrix}
		1&2&\ldots&n\\
		\alpha_1&\alpha_2&\ldots&\alpha_n
		\end{pmatrix}$} равен $(-1)^a$, где $a$ -- число инверсий в строке $(\alpha_1\ \alpha_2\ \ldots\ \alpha_n)$. 	Если знак равен 1, то подстановка четна, если -1 -- нечетна.
		
		$\;$
		\setlength{\parindent}{0cm}
		\hangindent=0cm
	}
	
	\textbf{7. Выписать общую формулу для вычисления определителя произвольного порядка} 
	
	{
		$\;$
		\setlength{\parindent}{0.4cm}
		\hangindent=0.4cm
		
		$\det A=\sum\limits_{\sigma\in S_n} sgn(\sigma) a_{1\sigma(1)}\cdot a_{2\sigma(2)}\cdot\ldots\cdot a_{n\sigma(n)}$ (сумма по всем подстановкам).
		
		$\;$
		\setlength{\parindent}{0cm}
		\hangindent=0cm
	}
	
	\textbf{8. Выписать формулы для разложения определителя по строке и столбцу.} 
	
	{
		$\;$
		\setlength{\parindent}{0.4cm}
		\hangindent=0.4cm
		
		Определитель матрицы  A  равен сумме произведений элементов $i$-той строки ($j$-того столбца) на их алгебраические дополнения:
	$$\det A=\sum\limits_{j=1}^n a_{ij}\cdot A_{ij}=\sum\limits_{i=1}^n a_{ij}\cdot A_{ij}$$
		
		$\;$
		\setlength{\parindent}{0cm}
		\hangindent=0cm
	}
	
	\newpage
	
	\textbf{9. Что такое фальшивое разложение?} 
	
	{
		$\;$
		\setlength{\parindent}{0.4cm}
		\hangindent=0.4cm
		
		Элементы строки при умножении на алгебраические дополнения к элементу другой строки дают после суммирования 0.
		
		$$\sum\limits_{j=1}^n a_{ij}\cdot A_{kj}=0,\text{ если }k\ne i$$
		$$\sum\limits_{i=1}^n a_{ij}\cdot A_{ik}=0,\text{ если }k\ne j$$
		
		$\;$
		\setlength{\parindent}{0cm}
		\hangindent=0cm
	}

	\textbf{10. Выписать формулы Крамера для квадратной матрицы произвольного порядка.}
	
	{
		$\;$
		\setlength{\parindent}{0.4cm}
		\hangindent=0.4cm
		
		Пусть $A\cdot x=b$ -- совместная СЛАУ. Тогда $\triangle_j = x_j\cdot\det(A_1,\ldots, A_n)=\det(A_1, \ldots, A_{j-1}, b, A_{j+1}, \ldots, A_n)$
		
		Если $\triangle\equiv\det A\ne 0$, то
		$x_j=\frac{\triangle_j}{\triangle}, \; j=\overline{1, n}$\\
		
		$\;$
		\setlength{\parindent}{0cm}
		\hangindent=0cm
	}
	
	\textbf{11. Что такое дополняющий минор и что такое алгебраическое дополнение?} 
	
	{
		$\;$
		\setlength{\parindent}{0.4cm}
		\hangindent=0.4cm
		
		В матрице $A_{n\times n}$ вычеркнем $i$-тую строку и $j$-тый столбец. Определитель получившейся матрицы называется \textit{дополняющим минором} элемента $a_{ij}$.
		
		$\;$
		
		\textit{Алгебраическим дополнением} элемента $a_{ij}$ называется число $(-1)^{i+j}\cdot M_{ij}=A_{ij}$\\
		
		$\;$
		\setlength{\parindent}{0cm}
		\hangindent=0cm
	}
	
	\textbf{12. Дать определение союзной матрицы.} 
	
	{
		$\;$
		\setlength{\parindent}{0.4cm}
		\hangindent=0.4cm
		
		\textit{Союзная матрица} -- транспонированная матрица из алгебраических	дополнений к элементам матрицы $A$.
		$$\tilde A=\begin{pmatrix}
		A_{11}&\ldots&A_{n1}\\
		\vdots&\ddots&\vdots\\
		A_{1n}&\ldots&A_{nn}
		\end{pmatrix}$$
		
		$\;$
		\setlength{\parindent}{0cm}
		\hangindent=0cm
	}

	\textbf{13. Дать определение обратной матрицы. Сформулировать критерий ее существования.} 
	
	{
		$\;$
		\setlength{\parindent}{0.4cm}
		\hangindent=0.4cm
			
		Матрица $B\in M_n(\mathbb{R})$ называется \textit{обратной} к матрице $A$, если $B\cdot A=E=A\cdot B$.
		
		$\;$
		
		Матрица $A\in M_n(\mathbb{R})$ имеет обратную (обратима) $\Leftrightarrow\det A\ne0$ (она невырождена).
		
		$\;$
		\setlength{\parindent}{0cm}
		\hangindent=0cm
	}
	
	\textbf{14. Выписать формулу для нахождения обратной матрицы.}
	
	{
		$\;$
		\setlength{\parindent}{0.4cm}
		\hangindent=0.4cm
		
		$A^{-1}=\dfrac{1}{\det A}\cdot \tilde A$, где $\tilde A$ -- союзная матрица.
		
		$\;$
		\setlength{\parindent}{0cm}
		\hangindent=0cm
	}
	
	\textbf{15. Дать определение минора.}
	
	{
		$\;$
		\setlength{\parindent}{0.4cm}
		\hangindent=0.4cm
		
		Минором $k$-го порядка матрицы $A$ называют определитель матрицы, составленной из элементов, стоящих на пересечениях произвольных $k$ строк и $k$ столбцов.\\
		
		$\;$
		\setlength{\parindent}{0cm}
		\hangindent=0cm
	}
	
	\textbf{16. Дать определение базисного минора. Какие строки называются базисными?}
	
	{
		$\;$
		\setlength{\parindent}{0.4cm}
		\hangindent=0.4cm
		
		Любой отличный от нуля минор, порядок которого равен рангу, называется \textit{базисным	минором} матрицы.
		
		$\;$
		
		Строки, попавшие в фиксированный базисный минор, называются \textit{базисными}.
		
		$\;$
		\setlength{\parindent}{0cm}
		\hangindent=0cm
	}
	
	\textbf{17. Дать определение ранга матрицы.} 
	
	{
		$\;$
		\setlength{\parindent}{0.4cm}
		\hangindent=0.4cm
		
			Рангом матрицы называют наибольший порядок отличного от 0 минора.
		
		$\;$
		\setlength{\parindent}{0cm}
		\hangindent=0cm
	}

	\newpage

	\textbf{18. Дать определение линейной комбинации строк. Что такое нетривиальная линейная комбинация?} 
	
	{
		$\;$
		\setlength{\parindent}{0.4cm}
		\hangindent=0.4cm
		
		\textit{Линейной комбинацией строк (столбцов)} $a_1, \ldots, a_s$ одинаковой длины (высоты) называют выражение вида $\lambda_1\cdot a_1+\ldots+\lambda_s\cdot a_s$, где $\lambda_1, \ldots, \lambda_s$ -- некоторые числа. 
		
		$\;$
		
		Линейная комбинация называется \textit{нетривиальной}, если $\exists \lambda_i\ne 0$.
		
		$\;$
		\setlength{\parindent}{0cm}
		\hangindent=0cm
	}
	
	\textbf{19. Дать определение линейной зависимости строк матрицы.} 
	
	{
		$\;$
		\setlength{\parindent}{0.4cm}
		\hangindent=0.4cm
		
		Строки $a_1, \ldots, a_s$ называют \textit{линейно зависимыми}, если существует нетривиальная линейная комбинация $\lambda_1\cdot a_1+\ldots+\lambda_s\cdot a_s=0$.
		
		$\;$
		\setlength{\parindent}{0cm}
		\hangindent=0cm
	}

	\textbf{20. Дать определение линейно независимых столбцов матрицы.} 
	
	{
		$\;$
		\setlength{\parindent}{0.4cm}
		\hangindent=0.4cm
		
		Если равенство $\lambda_1\cdot a_1+\ldots+\lambda_k\cdot a_k=0$ возможно только при $\lambda_1=\lambda_2=\ldots=\lambda_k=0$, то говорят, что столбцы $a_1, \ldots, a_k$ \textit{линейно независимы} (л.н.з.). 
		
		$\;$
		\setlength{\parindent}{0cm}
		\hangindent=0cm
	}
	
	\textbf{21. Сформулировать критерий линейной зависимости.} 
	
	{
		$\;$
		\setlength{\parindent}{0.4cm}
		\hangindent=0.4cm
		
		Строки $a_1, \ldots, a_k$ линейно зависимы $\Leftrightarrow$ хотя бы одна из них является линейной комбинацией остальных.
		
		$\;$
		\setlength{\parindent}{0cm}
		\hangindent=0cm
	}
	
	\textbf{22. Сформулировать теорему о базисном миноре.} 
	
	{
		$\;$
		\setlength{\parindent}{0.4cm}
		\hangindent=0.4cm
		
		1$\left.\right)$ Базисные строки (столбцы), соответсвующие любому базисному минору $M$ матрицы $A$ л.н.з.
		
		2$\left.\right)$ Строки (столбцы) матрицы $A$, не входящие в $M$, являются линейными комбинациями базисных строк (столбцов).
		
		$\;$
		\setlength{\parindent}{0cm}
		\hangindent=0cm
	}
	
	\textbf{23. Сформулировать теорему о ранге матрицы.} 
	
	{
		$\;$
		\setlength{\parindent}{0.4cm}
		\hangindent=0.4cm
		
		Ранг матрицы равен максимальному числу ее л.н.з. строк (столбцов).
		
		$\;$
		\setlength{\parindent}{0cm}
		\hangindent=0cm
	}
	
	\textbf{24. Сформулировать критерий невырожденности квадратной матрицы.} 
	
	{
		$\;$
		\setlength{\parindent}{0.4cm}
		\hangindent=0.4cm
		
		Рассмотрим матрицу $A\in M_n(\mathbb{R})$. Следующие условия эквивалентны:
		
		1$\left.\right)$ $\det A\ne 0$
		
		2$\left.\right)$ $RgA=n$
		
		3$\left.\right)$ все строки $A$ л.н.з.\\
		
		$\;$
		\setlength{\parindent}{0cm}
		\hangindent=0cm
	}
	
	\Large
	\centering
	
	$\;$
	
	\textbf{2-й модуль}
	
	\flushleft
	\small
	
	\textbf{1. Сформулируйте теорему Кронекера-Капелли.} 
	
	{
		$\;$
		\setlength{\parindent}{0.4cm}
		\hangindent=0.4cm
		
		СЛАУ $A\cdot x=b$ совместна $\Leftrightarrow RgA=Rg(A|b)$.
		
		$\;$
		\setlength{\parindent}{0cm}
		\hangindent=0cm
	}
	
	\textbf{2. Сформулируйте критерий существования ненулевого решения однородной системы линейных уравнений с квадратной матрицей.} 
	
	{
		$\;$
		\setlength{\parindent}{0.4cm}
		\hangindent=0.4cm
		
		Однородная СЛАУ $A\cdot x=0$ имеет ненулевое решение $\Leftrightarrow$ Матрица $A$ вырождена, то есть $\det A=0$.
		
		$\;$
		\setlength{\parindent}{0cm}
		\hangindent=0cm
	}
	
	\textbf{3. Дайте определение фундаментальной системы решений (ФСР) однородной СЛАУ.}
	
	{
		$\;$
		\setlength{\parindent}{0.4cm}
		\hangindent=0.4cm
		
		Любые $n-r$ линейно независимых столбцов, являющихся решениями однородной СЛАУ $A\cdot x=0$, где $n$ -- число неизвестных, а $r=RgA$, называют \textit{фундаментальной системой решений} (ФСР) однородной СЛАУ $A\cdot x=0$.
		
		$\;$
		\setlength{\parindent}{0cm}
		\hangindent=0cm
	}
	
	\textbf{4. Сформулируйте теорему о структуре общего решения однородной СЛАУ.}
	
	{
		$\;$
		\setlength{\parindent}{0.4cm}
		\hangindent=0.4cm
		
		 Пусть $\Phi_1, \ldots, \Phi_k$ -- ФСР однородной СЛАУ $A\cdot x=0$. Тогда любое решение этой СЛАУ можно представить в виде
		$x=c_1\cdot\Phi_1+\ldots+c_k\cdot\Phi_k$, где $c_1,\ldots, c_k$ - некоторые постоянные.
		
		$\;$
		\setlength{\parindent}{0cm}
		\hangindent=0cm
	}
	
	\textbf{5. Сформулируйте теорему о структуре общего решения неоднородной системы линейных алгебраических уравнений.} 
	
	{
		$\;$
		\setlength{\parindent}{0.4cm}
		\hangindent=0.4cm
		
		Пусть известно частное решение $\tilde x$ СЛАУ $A\cdot x=b$. Тогда любое решение этой СЛАУ можно 
		представить в виде $x=\tilde x+c_1\cdot\Phi_1+\ldots+c_k\cdot\Phi_k$, где $\Phi_1,\ldots, \Phi_k$ -- ФСР соответствующей однородной СЛАУ, а $c_1, \ldots, c_k$ -- некоторые постоянные.
		
		$\;$
		\setlength{\parindent}{0cm}
		\hangindent=0cm
	}
	
	\textbf{6. Что такое алгебраическая и тригонометрическая форма записи комплексного числа?}
	
	{
		$\;$
		\setlength{\parindent}{0.4cm}
		\hangindent=0.4cm
		
		Пусть $z \in \mathbb{C}$. Тогда:
		\begin{itemize}
		
		
		\item $z=x+iy$ -- \textit{алгебраическая} \textit{форма} записи, где $x, y\in\mathbb{R}$
		
		\item $z=r(\cos\varphi+i\sin\varphi)$ -- \textit{тригонометрическая} \textit{форма} записи, где $r=|z|=\sqrt{x^2+y^2},$ $ \cos\varphi=\dfrac{x}{r}, \; \sin\varphi=\dfrac{y}{r}$\\
		\end{itemize}
		
		$\;$
		\setlength{\parindent}{0cm}
		\hangindent=0cm
	}
	
	\textbf{7. Дайте определение модуля и аргумента комплексного числа. Что такое главное значение аргумента комплексного числа?} 
	
	{
		$\;$
		\setlength{\parindent}{0.4cm}
		\hangindent=0.4cm
		
		\textit{Модуль комплексного числа} $r=|z|=\sqrt{x^2+y^2}$.
		
		$\;$
		
		 \textit{Аргумент комплексного числа} -- угол между положительным направлением вещественной оси и радиус-вектором этой точки:
		$$\phi=Arg z=\arg z+2\pi k, \; k\in \mathbb{Z}.$$
		$ \arg z\in[\left. 0, 2\pi\right.) \text{ или} \arg z\in (\left.-\pi, \pi \right. ] - \text{главное значение аргумента}.$
		
		$\;$
		\setlength{\parindent}{0cm}
		\hangindent=0cm
	}
	
	\textbf{8. Сложение, умножение комплексных чисел. Что происходит с аргументами и модулями комплексных чисел при умножении и делении?} 
	
	{
		$\;$
		\setlength{\parindent}{0.4cm}
		\hangindent=0.4cm
		
		\textbf{Сложение:} $ (x_1, y_1)+(x_2, y_2)=(x_1+x_2, y_1+y_2)$ 
		
		$\;$
		
		\textbf{Умножение:} $(x_1, y_1)\cdot(x_2, y_2)=(x_1\cdot x_2-y_1\cdot y_2, x_1\cdot y_2+x_2\cdot y_1)$. 
		
		$\;$
		
		При умножении модули комплексных чисел перемножаются, а аргументы складываются. Модуль частного двух комплексных чисел равен частному модулей, а аргумент -- разности аргументов делимого и делителя.
		
		$\;$
		\setlength{\parindent}{0cm}
		\hangindent=0cm
	}
	
	\textbf{9. Что такое комплексное сопряжение? Как можно делить комплексные числа в алгебраической форме?} 
	
	{
		$\;$
		\setlength{\parindent}{0.4cm}
		\hangindent=0.4cm
		
		\textbf{Комплексное сопряжение:} $\overline{z}=\overline{a+b\cdot i}=a-b\cdot i$
		
		$\;$
		
		Пусть $z_1, z_2 \in \mathbb{C}$ и $z_2 \ne 0$. Тогда:
		$$\dfrac{z_1}{z_2}=\dfrac{z_1\cdot\overline{z_2}}{z_2\cdot\overline{z_2}} = \dfrac{z_1\cdot\overline{z_2}}{|z_2|^2}$$
		
		$\;$
		\setlength{\parindent}{0cm}
		\hangindent=0cm
	}
	
	\textbf{10. Выпишите формулу Муавра.} 
	
	{
		$\;$
		\setlength{\parindent}{0.4cm}
		\hangindent=0.4cm
		$$z^n=r^n(\cos n\phi+i\sin n\phi),\; n\in\mathbb{N}$$
		
		$\;$
		\setlength{\parindent}{0cm}
		\hangindent=0cm
	}
	
	\newpage
	
	\textbf{11. Как найти комплексные корни n-ой степени из комплексного числа? Сделайте эскиз, на котором отметьте исходное число и все корни из него.} 
	
	{
		$\;$
		\setlength{\parindent}{0.4cm}
		\hangindent=0.4cm
		
		Дано число $w=\rho\cdot(\cos\psi+i\cdot \sin\psi)$ и число $n\in\mathbb{N}$
		
		$$\sqrt[n]{w}=\left\{z=\sqrt[n]{\rho}\cdot\left(\cos\frac{\psi+2\pi k}{n}+i\cdot\sin\frac{\psi+2\pi k}{n}\right), \; k=\overline{0, n-1}\right\} $$
		
		$\sqrt[6]{1}:$ \\
		%\begin{figure}[h]
		%	\includegraphics[scale=0.7]{korni2.jpg}
		%\end{figure}
		\includegraphics[scale=0.7]{korni2.jpg}
		
		$\;$
		\setlength{\parindent}{0cm}
		\hangindent=0cm
	}
	
	\textbf{12. Сформулируйте основную теорему алгебры. Сформулируйте теорему Безу.}
	
	{
		$\;$
		\setlength{\parindent}{0.4cm}
		\hangindent=0.4cm
		
		\textbf{Основная теорема алгебры:} $\forall$ многочлена $f(z)=a_n\cdot z^n+a_{n-1}\cdot z^{n-1}+\ldots+a_0\cdot z^0, \; a_i\in\mathbb{C}, \; n\in\mathbb{N}, \; a_n\ne0$ $\exists$ корень $z_0\in\mathbb{C}$.
		
		$\;$
		
		\textbf{Теорема Безу:} Остаток от деления многочлена $f(x)$ на $x-c$ равен $f(c)$.
		
		$\;$
		\setlength{\parindent}{0cm}
		\hangindent=0cm
	}
	
	\textbf{13. Выпишите формулу Эйлера. Выпишите выражения для синуса и косинуса через экспоненту.} 
	
	{
		$\;$
		\setlength{\parindent}{0.4cm}
		\hangindent=0.4cm
		
		\textbf{Формула Эйлера:} $\cos\phi+i\cdot\sin\phi=e^{i\phi}, \; \phi\in\mathbb{R}$
		$$\cos\phi=\dfrac{e^{i\phi}+e^{-i\phi}}{2}, \; \sin\phi=\dfrac{e^{i\phi}-e^{-i\phi}}{2i}$$
		
		$\;$
		\setlength{\parindent}{0cm}
		\hangindent=0cm
	}
	
	\textbf{14. Какие многочлены называются неприводимыми?} 	
	
	{
		$\;$
		\setlength{\parindent}{0.4cm}
		\hangindent=0.4cm
		
		Многочлен называется \textit{приводимым,} если $\exists$ нетривиальное разложение $f=g\cdot h$ и \textit{неприводимым} в противном случае.
		
		$\;$
		\setlength{\parindent}{0cm}
		\hangindent=0cm
	}
	
	\textbf{15. Сформулируйте утверждение о разложении многочленов на неприводимые множители над комплексными числами.} 
	
	{
		$\;$
		\setlength{\parindent}{0.4cm}
		\hangindent=0.4cm
		
		$\forall$ многочлен степени $n>0$ разлагается в произведение неприводимых многочленов.
		
		Комплексный многочлен степени $n$ разлагается в произведение:
		
        $P_n(z)=a_n\cdot(z-z_1)^{\alpha_1}\cdot \ldots\cdot(z-z_k)^{\alpha_k}$, где сумма кратностей $\alpha_1+\ldots+\alpha_k=n, z_i\in\mathbb{C}$\\
		
		$\;$
		\setlength{\parindent}{0cm}
		\hangindent=0cm
	}
	
	\textbf{16. Дайте определение векторного произведения векторов в трехмерном пространстве.} 
	
	{
		$\;$
		\setlength{\parindent}{0.4cm}
		\hangindent=0.4cm
		
		Вектор $\overrightarrow{c}$ называют \textit{векторным произведением} векторов $\overrightarrow{a}$ и $\overrightarrow{b}$, если:
		
		1) $|\overrightarrow{c}|=|\overrightarrow{a}|\cdot|\overrightarrow{b}|\cdot\sin\varphi$, где $\varphi$ - угол между $\overrightarrow{a}$ и $\overrightarrow{b}$
		
		2) $\overrightarrow{c}\perp\overrightarrow{a}, \overrightarrow{c}\perp\overrightarrow{b}$
		
		3) тройка $\overrightarrow{a}, \overrightarrow{b}, \overrightarrow{c}$ -- правая
		
		$\;$
		\setlength{\parindent}{0cm}
		\hangindent=0cm
	}
	
	\newpage
	
	\textbf{17. Сформулируйте три алгебраических свойства векторного произведения.} 
	
	{
		$\;$
		\setlength{\parindent}{0.4cm}
		\hangindent=0.4cm
		
		1) $\overrightarrow{a}\times\overrightarrow{b}=-\overrightarrow{b}\times\overrightarrow{a}$ (\textit{антикоммутативность})
		
		2) $(\lambda\overrightarrow{a})\times\overrightarrow{b}=\lambda(\overrightarrow{a}\times\overrightarrow{b})$
		
		3) $(\overrightarrow{a}+\overrightarrow{b})\times\overrightarrow{c}=\overrightarrow{a}\times\overrightarrow{c}+\overrightarrow{b}\times\overrightarrow{c}$ (\textit{дистрибутивность})
		
		$\;$
		\setlength{\parindent}{0cm}
		\hangindent=0cm
	}
	
	\textbf{18. Выпишите формулу для вычисления векторного произведения в координатах, заданных в ортонормированном базисе.} 
	
	{
		$\;$
		\setlength{\parindent}{0.4cm}
		\hangindent=0.4cm
		
		Пусть $\overrightarrow{i}, \overrightarrow{j}, \overrightarrow{k}$ -- правый ортонормированный базис, $\overrightarrow{a}=a_x\overrightarrow{i}+a_y\overrightarrow{j}+a_z\overrightarrow{k}, \; \overrightarrow{b}=b_x\overrightarrow{i}+b_y\overrightarrow{j}+b_z\overrightarrow{k}$. Тогда: $$\overrightarrow{a}\times\overrightarrow{b}=\begin{vmatrix}
		\overrightarrow{i}&\overrightarrow{j}&\overrightarrow{k}\\
		a_x&a_y&a_z\\
		b_x&b_y&b_z
		\end{vmatrix}=\overrightarrow{i}(a_yb_z-b_ya_z)+\overrightarrow{j}(a_zb_x-a_xb_z)+\overrightarrow{k}(a_xb_y-a_yb_x)$$
		
		$\;$
		\setlength{\parindent}{0cm}
		\hangindent=0cm
	}
	
	\textbf{19. Сформулируйте критерий коллинеарности двух векторов с помощью векторного произведения.} 
	
	{
		$\;$
		\setlength{\parindent}{0.4cm}
		\hangindent=0.4cm
		
		Векторы $\overrightarrow{a}$ и $\overrightarrow{b}$ коллинеарны $ \Leftrightarrow$ $\overrightarrow{a}\times\overrightarrow{b}=\overrightarrow{0}$.
		
		$\;$
		\setlength{\parindent}{0cm}
		\hangindent=0cm
	}
	
	\textbf{20. Дайте определение смешанного произведения векторов. Как вычислить объем тетраэдра с помощью смешанного произведения?} 
	
	{
		$\;$
		\setlength{\parindent}{0.4cm}
		\hangindent=0.4cm
		
		\textit{Смешанным произведением векторов} $\overrightarrow{a}, \overrightarrow{b}, \overrightarrow{c}$ называют число $(\overrightarrow{a}\times\overrightarrow{b}, \overrightarrow{c})$. 
		
		$\;$
		
		Объем тетраэдра, построенного на векторах $\overrightarrow{a}, \overrightarrow{b}, \overrightarrow{c}$ равен $V_T=\frac{1}{6}|\langle\overrightarrow{a}, \overrightarrow{b},\overrightarrow{c}\rangle|$.
		
		$\;$
		\setlength{\parindent}{0cm}
		\hangindent=0cm
	}
	
	\textbf{21. Выпишите формулу для вычисления смешанного произведения в координатах, заданных в ортонормированном базисе.}
	
	{
		$\;$
		\setlength{\parindent}{0.4cm}
		\hangindent=0.4cm
		
		Пусть $\overrightarrow{i}, \overrightarrow{j}, \overrightarrow{k}$ -- правый ортонормированный базис, $\overrightarrow{a}=a_x\overrightarrow{i}+a_y\overrightarrow{j}+a_z\overrightarrow{k},  \overrightarrow{b}=b_x\overrightarrow{i}+b_y\overrightarrow{j}+b_z\overrightarrow{k}, \overrightarrow{c}=c_x\overrightarrow{i}+c_y\overrightarrow{j}+c_z\overrightarrow{k}$. Тогда: 
		$$\langle\overrightarrow{a}, \overrightarrow{b},\overrightarrow{c}\rangle=\begin{vmatrix}
		a_x&a_y&a_z\\
		b_x&b_y&b_z\\
		c_x&c_y&c_z
		\end{vmatrix}$$
		
		$\;$
		\setlength{\parindent}{0cm}
		\hangindent=0cm
	}
	
	\textbf{22. Сформулируйте критерий компланарности трех векторов с помощью смешанного произведения.}
	
	{
		$\;$
		\setlength{\parindent}{0.4cm}
		\hangindent=0.4cm
		
		Векторы $\overrightarrow{a}, \overrightarrow{b}, \overrightarrow{c}$ компланарны $\Leftrightarrow$ $\langle\overrightarrow{a}, \overrightarrow{b},\overrightarrow{c}\rangle=0$.
		
		$\;$
		\setlength{\parindent}{0cm}
		\hangindent=0cm
	}
	
	\textbf{23. Дайте определение прямоугольной декартовой системы координат.}
	
	{
		$\;$
		\setlength{\parindent}{0.4cm}
		\hangindent=0.4cm
		
		\textit{Прямоугольной декартовой системой координат} называют пару, состоящую из точки $O$ и ортонормированного базиса.
		
		$\;$
		\setlength{\parindent}{0cm}
		\hangindent=0cm
	}
	
	\textbf{24. Что такое уравнение поверхности и его геометрический образ?}
	
	{
		$\;$
		\setlength{\parindent}{0.4cm}
		\hangindent=0.4cm
		
		Уравнение $F(x, y, z)=0$ называют \textit{уравнением поверхности} $S$, если этому уравнению удовлетворяют координаты любой точки, лежащей на поверхности, и не удовлетворяют координаты ни одной точки, не лежащей на поверхности.
		
		$\;$
		
		При этом поверхность $S$ называют \textit{геометрическим образом} уравнения $F(x, y, z)=0$.
		
		$\;$
		\setlength{\parindent}{0cm}
		\hangindent=0cm
	}
	
	\textbf{25. Сформулируйте теорему о том, что задает любое линейное уравнение на координаты точки в трехмерном пространстве.}
	
	{
		$\;$
		\setlength{\parindent}{0.4cm}
		\hangindent=0.4cm
		
		 Любое уравнение $Ax+By+Cz+D=0$, где $A^2+B^2+C^2>0$, определяет в пространстве плоскость.
		
		$\;$
		\setlength{\parindent}{0cm}
		\hangindent=0cm
	}
	
	\textbf{26. Что такое нормаль к плоскости?}
	
	{
		$\;$
		\setlength{\parindent}{0.4cm}
		\hangindent=0.4cm
		
		Пусть $Ax+By+Cz+D=0$ -- уравнение плоскости. Тогда вектор $\overrightarrow{n}=(A, B, C)$ 
		перпендикулярен плоскости и называется нормалью к этой плоскости.
		
		$\;$
		\setlength{\parindent}{0cm}
		\hangindent=0cm
	}
	
	\textbf{27. Выпишите формулу расстояния от точки до плоскости.}
	
	{
		$\;$
		\setlength{\parindent}{0.4cm}
		\hangindent=0.4cm
		
		Рассмотрим плоскость $L:Ax+By+Cz+D=0$ и точку $M(x_0, y_0, z_0)$. Тогда:
		$$\rho(M, L)=\dfrac{|Ax_0+By_0+Cz_0+D|}{\sqrt{A^2+B^2+C^2}}$$
		
		$\;$
		\setlength{\parindent}{0cm}
		\hangindent=0cm
	}
	
	\textbf{28. Общие уравнения прямой. Векторное уравнение прямой. Параметрические и канонические уравнения прямой.}
	
	{
		$\;$
		\setlength{\parindent}{0.4cm}
		\hangindent=0.4cm
		
		\begin{itemize}
		\item $\begin{cases}
		A_1x+B_1y+C_1z+D_1=0\\
		A_2x+B_2y+C_2z+D_2=0
		\end{cases}$-- общее уравнение прямой
		
		\item Векторное уравнение прямой: $\overrightarrow{r}=\overrightarrow{r_0}+t\overrightarrow{s}$, где $\overrightarrow{r_0}$ -- радиус-вектор некоторой точки прямой, $\overrightarrow{s}$ -- направляющий вектор прямой
		
	  	\item Параметрическое уравнение: $\left\lbrace \begin{aligned}
		&x-x_0=tl\\
		&y-y_0=tm\\
		&z-z_0=tn
		\end{aligned}\right. $, где $\overrightarrow{p}(l, m, n)$ -- направляющий вектор прямой, 
		
		$M(x_0, y_0, z_0)$ -- точка прямой
		
		\item Каноническое уравнение прямой: $t=\dfrac{x-x_0}{l}=\dfrac{y-y_0}{m}=\dfrac{z-z_0}{n}$\\
		
		\end{itemize}
		$\;$
		\setlength{\parindent}{0cm}
		\hangindent=0cm
	}
	
	\textbf{29. Сформулируйте критерий принадлежности двух прямых одной плоскости.}
	
	{
		$\;$
		\setlength{\parindent}{0.4cm}
		\hangindent=0.4cm
		
		Пусть $M_1(x_1, y_1, z_1)\in L_1, \; M_2(x_2, y_2, z_2)\in L_2$. Тогда $L_1$ и $L_2$ в одной плоскости $\Leftrightarrow\overrightarrow{s_1}, \overrightarrow{s_2}$ и $\overrightarrow{M_1M_2}$ компланарны, где $\overrightarrow{s_1}, \overrightarrow{s_2}$ -- направляющие вектора прямых $L_1$ и $L_2$ соответственно.
		
		$\;$
		\setlength{\parindent}{0cm}
		\hangindent=0cm
	}
	
	\textbf{30. Выпишите формулу для вычисления расстояния от точки до прямой.}
	
	{
		$\;$
		\setlength{\parindent}{0.4cm}
		\hangindent=0.4cm
		
		Рассмотрим точку $M_1(x_1, y_1, z_1)$ и прямую $L:\dfrac{x-x_0}{l}=\dfrac{y-y_0}{m}=\dfrac{z-z_0}{n}$. Пусть $\overrightarrow{s}=(l, m, n),\; M_0(x_0, y_0, z_0)$. Тогда: $$\rho(M_1, L)=\dfrac{|\overrightarrow{M_0M_1}\times\overrightarrow{s}|}{|\overrightarrow{s}|}$$
		
		$\;$
		\setlength{\parindent}{0cm}
		\hangindent=0cm
	}
	
	\textbf{31. Выпишите формулу для вычисления расстояния между двумя скрещивающимися прямыми.}
	
	{
		$\;$
		\setlength{\parindent}{0.4cm}
		\hangindent=0.4cm
		
		Рассмотрим скрещивающиеся прямые $L_1$ и $L_2$, $s_1$ и $s_2$ -- их направляющие векторы и точки $M_1\in L_1, \; M_2\in L_2$. Тогда: $$\rho(L_1, L_2)=\dfrac{|\langle\overrightarrow{s_1}, \overrightarrow{s_2}, \overrightarrow{M_1M_2}\rangle|}{|\overrightarrow{s_1}\times\overrightarrow{s_2}|}$$
		
		$\;$
		\setlength{\parindent}{0cm}
		\hangindent=0cm
	}
	
	\textbf{32. Какие бинарные операции называются ассоциативными, а какие коммутативными?}
	
	{
		$\;$
		\setlength{\parindent}{0.4cm}
		\hangindent=0.4cm
		
		Бинарная операция $\times$ называется \textit{ассоциативной}, если $\forall a, b, c\in X:$ $a\times(b\times c)=(a\times b)\times c$.
		
		$\;$
		
		Бинарная операция $\ast$ называется \textit{коммутативной}, если $\forall a, b\in X$ $a\ast b=b\ast a$.
		
		$\;$
		\setlength{\parindent}{0cm}
		\hangindent=0cm
	}
	
	\newpage
	
	\textbf{33. Дайте определение полугруппы и моноида. Приведите примеры.}
	
	{
		$\;$
		\setlength{\parindent}{0.4cm}
		\hangindent=0.4cm
		
		Множество с заданной на нем ассоциативной бинарной операцией называется \textit{полугруппой}. \textbf{Пример:} $(\mathbb{N}, +)$.
				
		$\;$
		
		Полугруппа, в которой есть нейтральный элемент, называется \textit{моноидом}. \textbf{Пример:} $(\mathbb{N}, \cdot)$ -- моноид, $e=1$.
		
		$\;$
		\setlength{\parindent}{0cm}
		\hangindent=0cm
	}
	
	\textbf{34. Сформулируйте определение группы. Приведите пример.}
	
	{
		$\;$
		\setlength{\parindent}{0.4cm}
		\hangindent=0.4cm
		
		Моноид $G$, все элементы которого обратимы, называется \textit{группой}. \textbf{Пример:} множество всех невырожденных $(\det A\ne 0)$ матриц $A_{n\times n}$ с операцией матричного умножения.
		
		$\;$
		\setlength{\parindent}{0cm}
		\hangindent=0cm
	}
	
	\textbf{35. Что такое симметрическая группа? Укажите число элементов в ней.}
	
	{
		$\;$
		\setlength{\parindent}{0.4cm}
		\hangindent=0.4cm
		
		\textit{Симметрическая группа} $S_n$ -- множество всех подстановок длины $n$ {\scriptsize $\sigma=\begin{pmatrix}
		1&\ldots&n\\
		l_1&\ldots&l_n
		\end{pmatrix}$} с операцией композиции. В ней $n!$ элементов.		
	
		$\;$
		\setlength{\parindent}{0cm}
		\hangindent=0cm
	}
	
	\textbf{36. Что такое общая линейная и специальная линейная группы?}
	
	{
		$\;$
		\setlength{\parindent}{0.4cm}
		\hangindent=0.4cm
		
		Множество всех невырожденных $(\det A\ne 0)$ матриц $A_{n\times n}$ с операцией матричного умножения -- $GL_n(\mathbb{R})$ -- \textit{общая линейная  группа}.
		
		$\;$
		
		$SL_n(\mathbb{R})=\{A\in GL_n(\mathbb{R})|\det A=1 \}$ -- \textit{специальная линейная группа}.	
		
		$\;$
		\setlength{\parindent}{0cm}
		\hangindent=0cm
	}
	
	\textbf{37. Сформулируйте определение абелевой группы. Приведите пример.}
	
	{
		$\;$
		\setlength{\parindent}{0.4cm}
		\hangindent=0.4cm
		
		Группа с коммутативной операцией называется \textit{абелевой}. \textbf{Пример:} $(\mathbb{Z}, +)$ -- абелева группа.
		
		$\;$
		\setlength{\parindent}{0cm}
		\hangindent=0cm
	}
	
	\textbf{38. Дайте определение подгруппы. Приведите пример группы и ее подгруппы.}
	
	{
		$\;$
		\setlength{\parindent}{0.4cm}
		\hangindent=0.4cm
		
		
		Подмножество $H\subseteq G$ называется \textit{подгруппой} в группе $G$, если:
		
		1) $e\in H$
		
		2) $\forall h_1, h_2\in H: h_1\cdot h_2\in H$
		
		3) $\forall h\in H\Rightarrow h^{-1}\in H$
		
		$\;$
		
		\textbf{Пример:} $SL_n(\mathbb{R})\subset GL_n(\mathbb{R})$\\
		
		$\;$
		\setlength{\parindent}{0cm}
		\hangindent=0cm
	}
	
	\textbf{39. Дайте определение гомоморфизма групп. Приведите пример.}
	
	{
		$\;$
		\setlength{\parindent}{0.4cm}
		\hangindent=0.4cm
		
		
		Отображение $f:G\rightarrow G'$ группы $(G, \ast)$ в группу $(G', \circ)$ называется \textit{гомоморфизмом}, если $\forall a, b\in G$ $f(a\ast b)=f(a)\circ f(b)$.
		
		$\;$
		
		\textbf{Пример:} $\det:$ $GL_n(\mathbb{R})\rightarrow\mathbb{R}^{\ast}$ ($\mathbb{R}^{\ast}$ -- это $\mathbb{R}\backslash\{0\}$ с операцией умножения). Это гомоморфизм, так как $\det(A\cdot B)=\det A\cdot \det B$.
		
		$\;$
		\setlength{\parindent}{0cm}
		\hangindent=0cm
	}
	
	\textbf{40. Дайте определение изоморфизма групп. Приведите пример.}
	
	{
		$\;$
		\setlength{\parindent}{0.4cm}
		\hangindent=0.4cm
		
		
		\textit{Изоморфизм} -- это биективный гомоморфизм.
		
		$\;$
		
		\textbf{Пример:} $(\mathbb{R}, +)\simeq(\mathbb{R}^+, \cdot)$ посредством изоморфизма $f(x)=e^x$.
		
		$\;$
		\setlength{\parindent}{0cm}
		\hangindent=0cm
	}
	
	\textbf{41. Дайте определение порядка элемента}
	
	{
		$\;$
		\setlength{\parindent}{0.4cm}
		\hangindent=0.4cm
		
		
		\textit{Порядок элемента} $a\in G$ -- наименьшее натуральное число $p$ такое, что $a^p=e$.
		
		$\;$
		\setlength{\parindent}{0cm}
		\hangindent=0cm
	}
	
	\Large
	\centering
	
	$\;$
	
	\textbf{3-й модуль}
	
	\flushleft
	\small
	
	\textbf{1. Что такое ядро гомоморфизма групп? Приведите пример.} 
	
	{
		$\;$
		\setlength{\parindent}{0.4cm}
		\hangindent=0.4cm
		
		
		\textit{Ядро гомоморфизма} $f:G\rightarrow F$ $Ker f=\{g\in G|f(g)=e_F \}$ ($e_F$ -- нейтральный элелемент в $F$).
		
		$\;$
		
		\textbf{Пример:} В гомоморфизме $\mathbb{Z}\rightarrow\mathbb{Z}/3\mathbb{Z}$ с $h(u)=u \mod 3$ ядро состоит из целых чисел, делящихся на 3.
		
		$\;$
		\setlength{\parindent}{0cm}
		\hangindent=0cm
	}
	
	\textbf{2. Сформулируйте определение циклической группы. Приведите пример.}
	
	{
		$\;$
		\setlength{\parindent}{0.4cm}
		\hangindent=0.4cm
		
		Если $\forall$ элемент $g\in G$ имеет вид $g=a^n=a\times a\times\ldots\times a$ ($n$ раз), где $a\in G$, то $G$ -- \textit{циклическая группа}. 
		
		$\;$
		
		\textbf{Пример:} $(\mathbb{Z}, +)$ -- циклическая группа, порожденная 1.
		
		$\;$
		\setlength{\parindent}{0cm}
		\hangindent=0cm
	}
	
	\textbf{3. Сколько существует, с точностью до изоморфизма, циклических групп данного порядка?}
	
	{
		$\;$
		\setlength{\parindent}{0.4cm}
		\hangindent=0.4cm
		
		Существует ровно одна циклическая группа данного порядка с точностью до изоморфизма.
		
		$\;$
		\setlength{\parindent}{0cm}
		\hangindent=0cm
	}
	
	\textbf{4. Что такое группа диэдра? Что такое знакопеременная группа? Укажите число элементов в них.}
	
	{
		$\;$
		\setlength{\parindent}{0.4cm}
		\hangindent=0.4cm
		
		\textit{Группа диэдра} ($D_n$) -- это группа симметрии правильного $n$-угольника, $|D_n|=2n$.
		
		$\;$
		
		$A_n$ -- \textit{знакопеременная группа}, то есть множество всех четных подстановок, $|A_n|=\dfrac{n!}{2}$.
		
		$\;$
		\setlength{\parindent}{0cm}
		\hangindent=0cm
	}
	
	\textbf{5. Сформулируйте утверждение о связи порядка элемента, порождающего циклическую группу, с порядком группы.}
	
	{
		$\;$
		\setlength{\parindent}{0.4cm}
		\hangindent=0.4cm
		
		Пусть $G$ -- группа и $g\in G$, тогда $ord(g)=|\langle g\rangle|$.
		
		$\;$
		\setlength{\parindent}{0cm}
		\hangindent=0cm
	}
	
	\textbf{6. Сформулируйте утверждение о том, какими могут быть подгруппы группы целых чисел по сложению.}
	
	{
		$\;$
		\setlength{\parindent}{0.4cm}
		\hangindent=0.4cm
		
		 $\forall$ подгруппа в $(\mathbb{Z}, +)$ имеет вид $k\mathbb{Z}$ для некоторых $k\in \mathbb{N}\cup\{0\}$.
		
		$\;$
		\setlength{\parindent}{0cm}
		\hangindent=0cm
	}
	
	\textbf{7. Дайте определение левого смежного класса по некоторой подгруппе.}
	
	{
		$\;$
		\setlength{\parindent}{0.4cm}
		\hangindent=0.4cm
		
		Пусть $G$ -- группа, $H\subseteq G$ --  подгруппа и $g\in G$. Тогда \textit{левым смежным классом} элемента $g$ по подгруппе $H$ называется множество $gH=\{gh|h\in H \}$.
		
		$\;$
		\setlength{\parindent}{0cm}
		\hangindent=0cm
	}
	
	\textbf{8. Дайте определение нормальной подгруппы.}
	
	{
		$\;$
		\setlength{\parindent}{0.4cm}
		\hangindent=0.4cm
		
		Подгруппа $H$ называется \textit{нормальной}, если $gH=Hg$, $\forall g\in G$ (равенство правых и левых смежных классов).
		
		$\;$
		\setlength{\parindent}{0cm}
		\hangindent=0cm
	}
	
	\textbf{9. Что такое индекс подгруппы?}
	
	{
		$\;$
		\setlength{\parindent}{0.4cm}
		\hangindent=0.4cm
		
		\textit{Индексом} подгруппы $H$ в группе $G$ называется число левых смежных классов $G$ по $H$.
		
		$\;$
		\setlength{\parindent}{0cm}
		\hangindent=0cm
	}
	
	\textbf{10. Сформулируйте теорему Лагранжа.}
	
	{
		$\;$
		\setlength{\parindent}{0.4cm}
		\hangindent=0.4cm
		
		Пусть $G$ -- конечная группа и $H\subseteq G$ -- подгруппа. Тогда $|G|=|H|\cdot[G:H]$.
		
		$\;$
		\setlength{\parindent}{0cm}
		\hangindent=0cm
	}
	
	\textbf{11. Сформулируйте две леммы, которые нужны для доказательства теоремы Лагранжа.}
	
	{
		$\;$
		\setlength{\parindent}{0.4cm}
		\hangindent=0.4cm
		
		\textbf{Лемма 1:} $\forall g_1, g_2\in G$ либо $g_1H=g_2H$, либо $g_1H\cap g_2H=\O$.
		
		$\;$
		
		\textbf{Лемма 2:} $|gH|=|H|$ $\forall g\in G$ $, \forall$ конечной подгруппы $H$.
		
		$\;$
		\setlength{\parindent}{0cm}
		\hangindent=0cm
	}
	
	\textbf{12. Сформулируйте три следствия из теоремы Лагранжа.}
	
	{
		$\;$
		\setlength{\parindent}{0.4cm}
		\hangindent=0.4cm
		
		\textbf{Следствие 1:} Пусть $G$ -- конечная группа и $g\in G$. Тогда $ord(g)$ делит $|G|$.
		
		$\;$
		
		\textbf{Следствие 2:} Пусть $G$ -- конечная группа. Тогда $g^{|G|}=e$.
		
		$\;$
		
		\textbf{Следствие 3 (малая теорема Ферма):}
		Пусть $\overline a $ -- ненулевой вычет по простому модулю $p$. Тогда $\overline{a}^{p-1}\equiv1\mod p$.
		
		$\;$
		\setlength{\parindent}{0cm}
		\hangindent=0cm
	}
	
	\textbf{13. Сформулируйте критерий нормальности подгруппы, использующий сопряжение.}
	
	{
		$\;$
		\setlength{\parindent}{0.4cm}
		\hangindent=0.4cm
		
		Пусть $H\subseteq G$ -- подгруппа в группе $G$. Тогда 3 условия эквивалентны:
		
		1) $H$ нормальна
		
		2) $\forall g\in G$ $gHg^{-1}\subseteq H$ ($gHg^{-1}=\{ghg^{-1}|h\in H \}$)
		
		3) $\forall g\in G$ $gHg^{-1}=H$
		
		$\;$
		\setlength{\parindent}{0cm}
		\hangindent=0cm
	}
	
	\textbf{14. Дайте определение факторгруппы.}
	
	{
		$\;$
		\setlength{\parindent}{0.4cm}
		\hangindent=0.4cm
		
		Пусть $H$ -- нормальная подгруппа. Тогда $G/ H$ -- множество левых смежных классов по $H$ с операцией умножения: $(g_1H)\cdot(g_2H)=g_1\cdot g_2H$ называется \textit{факторгруппой} $G$ по $H$.
		
		$\;$
		\setlength{\parindent}{0cm}
		\hangindent=0cm
	}
	
	\textbf{15. Что такое естественный гомоморфизм?}
	
	{
		$\;$
		\setlength{\parindent}{0.4cm}
		\hangindent=0.4cm
		
		Отображение $\varepsilon:G\rightarrow G/H$, сопоставляющее каждому элементу $a\in G$ его класс смежности $aH$, называется \textit{естественным гомоморфизмом}.
		
		$\;$
		\setlength{\parindent}{0cm}
		\hangindent=0cm
	}
	
	 \textbf{16. Сформулируйте критерий нормальности подгруппы, использующий понятие ядра гомоморфизма.}
	
	{
		$\;$
		\setlength{\parindent}{0.4cm}
		\hangindent=0.4cm
		
		$H$ -- нормальная подгруппа $\Leftrightarrow H=Ker f$, где $f$ -- некоторый гомоморфизм.
		
		$\;$
		\setlength{\parindent}{0cm}
		\hangindent=0cm
	}
	
	\textbf{17. Сформулируйте теорему о гомоморфизме групп. Приведите пример.}
	
	{
		$\;$
		\setlength{\parindent}{0.4cm}
		\hangindent=0.4cm
		
		Пусть $f:G\rightarrow F$ -- гомоморфизм групп. Тогда группа $Im f =\{a\in F|\exists g\in G, f(g)=a \}$ изоморфна факторгруппе $G/Ker f$, $Ker f=\{g\in G|f(g)=e_F \}$ ($Ker f$ -- ядро гомоморфизма).
	$$G/Ker f\simeq Im f$$
		
		\textbf{Пример:} $\mathbb{Z}/n\mathbb{Z}\simeq\mathbb{Z}_n$ $f:\mathbb{Z}\rightarrow\mathbb{Z}_n$, $\forall$ целому числу сопоставляем его остаток от деления на $n$ -- $Ker f=n\mathbb{Z}$.
		
		$\;$
		\setlength{\parindent}{0cm}
		\hangindent=0cm
	}
	
	\textbf{18. Что такое прямое произведение групп?}
	
	{
		$\;$
		\setlength{\parindent}{0.4cm}
		\hangindent=0.4cm
		
		\textit{Прямое произведение групп} $(G, +)\times(D, \star)$ -- это группа из всех пар элементов с операцией поэлементного умножения: $$(g_1, d_1)\times(g_2, d_2)=(g_1+g_2, d_1\star d_2)$$.
		$\;$
		\setlength{\parindent}{0cm}
		\hangindent=0cm
	}
	
	\textbf{19. Сформулируйте определение автоморфизма и внутреннего автоморфизма.}
	
	{
		$\;$
		\setlength{\parindent}{0.4cm}
		\hangindent=0.4cm
		
		\textit{Автоморфизм} -- это изоморфизм из $G$ в $G$.
		
		$\;$
		
		\textit{Внутренний автоморфизм} -- это отображение $I_a:g\mapsto aga^{-1}$.
		
		$\;$
		\setlength{\parindent}{0cm}
		\hangindent=0cm
	}
	\textbf{}\\
	\textbf{}\\
	\textbf{}\\
	\textbf{}\\
	\textbf{20. Что такое центр группы? Что можно сказать о его свойствах?}
	
	{
		$\;$
		\setlength{\parindent}{0.4cm}
		\hangindent=0.4cm
		
		\textit{Центр} группы $G$ -- это множество $Z(G)=\{a\in G|ab=ba\;\forall b\in G \}$. $G$ -- абелева $\Leftrightarrow Z(G)=G$. $Z(G)$ является нормальной подгруппой $G$.
		
		$\;$
		\setlength{\parindent}{0cm}
		\hangindent=0cm
	}
	
	\textbf{21. Чему изоморфна факторгруппа группы по ее центру?}
	
	{
		$\;$
		\setlength{\parindent}{0.4cm}
		\hangindent=0.4cm
		
		$G/Z(G)\simeq Inn(G)$ ($Inn$ -- подгруппа, которую образуют все внутренние автоморфизмы группы $Aut(G)$).
		
		$\;$
		\setlength{\parindent}{0cm}
		\hangindent=0cm
	}
	
	\textbf{22. Сформулируйте теорему Кэли.}
	
	{
		$\;$
		\setlength{\parindent}{0.4cm}
		\hangindent=0.4cm
		
		$\forall$ конечная группа порядка $n$ изоморфна некоторой подгруппе группы $S_n$.
		
		$\;$
		\setlength{\parindent}{0cm}
		\hangindent=0cm
	}
	
	\textbf{23. Дайте определение кольца.}
	
	{
		$\;$
		\setlength{\parindent}{0.4cm}
		\hangindent=0.4cm
		
		Пусть $K\ne \O$ -- множество, на котором заданы две бинарные операции $"+"$ и $"\cdot"$, такие, что:
		
		1) $(K, +)$ -- абелева группа (это аддитивная группа кольца)
		
		2) $(K, \cdot)$ -- полугруппа (это мультипликативная полугруппа кольца)
		
		3) Умножение дистрибутивно относительно сложения: $\forall a, b, c\in K:$ $c(a+b)=ca+cb$, $(a+b)c=ac+bc$
		
		Тогда $(K, +, \cdot)$ -- \textit{кольцо}.
		
		$\;$
		\setlength{\parindent}{0cm}
		\hangindent=0cm
	}
	
	\textbf{24. Что такое коммутативное кольцо? приведите примеры коммутативного и некоммутативного колец.}
	
	{
		$\;$
		\setlength{\parindent}{0.4cm}
		\hangindent=0.4cm
		
		Если $\forall x, y\in K\;xy=yx$, то кольцо называется \textit{коммутативным}. 
		
		$\;$
		
		\textbf{Пример 1:} $(\mathbb{Z}, +, \cdot)$ -- является коммутативным кольцом.
		
		$\;$
		
		\textbf{Пример 2:} ($M_n(\mathbb{R}), +, \cdot$) -- полное матричное кольцо над $\mathbb{R}$ -- некоммутативное.
		
		$\;$
		\setlength{\parindent}{0cm}
		\hangindent=0cm
	}
	
	\textbf{25. Дайте определение делителей нуля.}
	
	{
		$\;$
		\setlength{\parindent}{0.4cm}
		\hangindent=0.4cm
		
		Если $a\cdot b=0$, при $a\ne 0, \; b\ne 0$ в кольце $K$, то $a$ называется \textit{левым делителем нуля}, а $b$ -- \textit{правым делителем нуля}.
		
		$\;$
		\setlength{\parindent}{0cm}
		\hangindent=0cm
	}
	
	\textbf{26. Дайте определение целостного кольца. Приведите пример.}
	
	{
		$\;$
		\setlength{\parindent}{0.4cm}
		\hangindent=0.4cm
		
		Коммутативное кольцо с единицей ($\ne 0$) и без делителей нуля называется \textit{целостным кольцом}. \textbf{Пример:} $(\mathbb{Z}, +, \cdot)$.
		
		
		$\;$
		\setlength{\parindent}{0cm}
		\hangindent=0cm
	}
	
	\textbf{27. Сформулируйте критерий целостности для нетривиального коммутативного кольца с единицей.}
	
	{
		$\;$
		\setlength{\parindent}{0.4cm}
		\hangindent=0.4cm
		
		Нетривиальное коммутативное кольцо с единицей является целостным $\Leftrightarrow$ в нем выполняется закон сокращения, то есть из $a\cdot b=a\cdot c$ при условии $a\ne 0\Rightarrow b=c\;\;\forall a, b, c\in K$.
		
		$\;$
		\setlength{\parindent}{0cm}
		\hangindent=0cm
	}
	
	\textbf{28. Какие элементы кольца называются обратимыми?}
	
	{
		$\;$
		\setlength{\parindent}{0.4cm}
		\hangindent=0.4cm
		
		Элемент коммутативного кольца $a$ называется \textit{обратимым}, если $\exists a^{-1}:a\cdot a^{-1}=1=a^{-1}\cdot a$.
		
		$\;$
		\setlength{\parindent}{0cm}
		\hangindent=0cm
	}
	
	\textbf{29. Дайте определение поля. Приведите три примера.}
	
	{
		$\;$
		\setlength{\parindent}{0.4cm}
		\hangindent=0.4cm
		
		\textit{Поле} $P$ -- это коммутативное кольцо с единицей ($\ne 0$), в котором каждый элемент $a\ne 0$ обратим. \textbf{Пример:} $\mathbb{R}, \mathbb{C}, \mathbb{Q}$.
		
		$\;$
		\setlength{\parindent}{0cm}
		\hangindent=0cm
	}
	
	\textbf{30. Дайте определение подполя. Привести пример пары: поле и его подполе.}
	
	{
		$\;$
		\setlength{\parindent}{0.4cm}
		\hangindent=0.4cm
		
		\textit{Подполе} -- это подмножество поля, которое само является полем относительно тех же операций. \textbf{Пример:} $\mathbb{Q}\subset\mathbb{R}$.
		
		$\;$
		\setlength{\parindent}{0cm}
		\hangindent=0cm
	}
	
	\newpage
	
	\textbf{31. Дайте определение характеристики поля. Привести примеры: поля конечной положительной характеристики и поля нулевой характеристики.}
	
	{
		$\;$
		\setlength{\parindent}{0.4cm}
		\hangindent=0.4cm
		
		Пусть $P$ -- поле. \textit{Характеристикой} поля $P$ ($char P$) называется наименьшее $q\in\mathbb{N}:$ $\underbrace{1+\ldots+1}_{q}=0$. Если такого $q$ не существует, то $char P=0$.
		
		$\;$
		
		\textbf{Пример:} $char\mathbb{R}=0$, $char\mathbb{Z}_p=p$, $p$ -- простое.
		
		$\;$
		\setlength{\parindent}{0cm}
		\hangindent=0cm
	}
	
	\textbf{32. Сформулируйте утверждение о том, каким будет простое подполе в зависимости от характеристики.}
	
	{
		$\;$
		\setlength{\parindent}{0.4cm}
		\hangindent=0.4cm
		
		Пусть $F$ -- поле. $F_0$ -- его простое подполе. Тогда:
		
		1) если $char F=p>0$, то $F_0\simeq \mathbb{Z}_p$
		
		2) если $char F=0$, то $F_0\simeq\mathbb{Q}$
		
		$\;$
		\setlength{\parindent}{0cm}
		\hangindent=0cm
	}
	
	\textbf{33. Дайте определение идеала. Что такое главный идеал?}
	
	{
		$\;$
		\setlength{\parindent}{0.4cm}
		\hangindent=0.4cm
		
		Подмножество $I$ кольца называется \textit{идеалом}, если:
		
		1. оно является подгруппой по сложению
		
		2. $\forall a\in I, \forall r\in K\;\; r\cdot a$ и $a\cdot r\in I$
		
		$\;$
		
		Идеал называется \textit{главным}, если $\exists a\in K:I=<a>$.
		
		$\;$
		\setlength{\parindent}{0cm}
		\hangindent=0cm
	}
	
	\textbf{34. Сформулируйте определение гомоморфизма колец.}
	
	{
		$\;$
		\setlength{\parindent}{0.4cm}
		\hangindent=0.4cm
		
		$\varphi:K_1\rightarrow K_2$ -- \textit{гомоморфизм колец}, если $\forall a, b\in K_1:\begin{cases}
		\varphi(a+b)=\varphi(a)\oplus\varphi(b)\\
		\varphi(a\cdot b)=\varphi(a)\ast\varphi(b)
		\end{cases}$
		
		$\;$
		\setlength{\parindent}{0cm}
		\hangindent=0cm
	}
	
	\textbf{35. Сформулируйте теорему о гомоморфизме колец. Приведите пример.}
	
	{
		$\;$
		\setlength{\parindent}{0.4cm}
		\hangindent=0.4cm
		
		Пусть $\varphi:K_1\rightarrow K_2$ -- гомоморфизм колец. Тогда $K_1/Ker\varphi\simeq Im\varphi$.
		
		$\;$
		
		\textbf{Пример:} $\mathbb{Z}/n\mathbb{Z}\simeq\mathbb{Z}_n$ $\varphi:\mathbb{Z}\rightarrow\mathbb{Z}_n$, $\forall$ целому числу сопоставляем его остаток от деления на $n$, $Ker \varphi=n\mathbb{Z}$.
		
		$\;$
		\setlength{\parindent}{0cm}
		\hangindent=0cm
	}
	
	\textbf{36. Сформулируйте теорему о том, когда факторкольцо кольца многочленов над полем само является полем.}
	
		{
		$\;$
		\setlength{\parindent}{0.4cm}
		\hangindent=0.4cm
		
		Факторкольцо $F[x]/<f(x)>$ является полем $\Leftrightarrow f(x)$ неприводим над $F$.
		
		$\;$
		\setlength{\parindent}{0cm}
		\hangindent=0cm
	}
	
	\textbf{37. Сформулируйте критерий того, что кольцо вычетов по модулю $p$ является полем.}
	
	{
		$\;$
		\setlength{\parindent}{0.4cm}
		\hangindent=0.4cm
		
		$\mathbb{Z}_p$ -- поле $\Leftrightarrow p$ -- простое.
		
		$\;$
		\setlength{\parindent}{0cm}
		\hangindent=0cm
	}
	
	 \textbf{38. Дайте определение алгебраического элемента над полем.}
	
	{
		$\;$
		\setlength{\parindent}{0.4cm}
		\hangindent=0.4cm
		
		Пусть $F_2$ - поле, а $F_1$ - его подполе. Элемент $\alpha\in F_2$ называется \textit{алгебраическим} над полем $F_1$, если $\exists f(x)\ne0$ (0 как функция), что $f(x)\in F_1[x]$, для которого $f(\alpha)=0$.
		
		$\;$
		\setlength{\parindent}{0cm}
		\hangindent=0cm
	}
	
	\textbf{39. Что такое поле рациональных дробей?}
	
	{
		$\;$
		\setlength{\parindent}{0.4cm}
		\hangindent=0.4cm
		
		Пусть $F$ -- поле. Рассмотрим поле рациональных функций (частных) с коэфициентами из $F$. То есть элементы этого множества -- дроби $\dfrac{f(x)}{g(x)}$, где $f, g\in F[x], \;g\ne 0$.
		
		$\;$
		\setlength{\parindent}{0cm}
		\hangindent=0cm
	}
	
	\newpage
	
	\textbf{40. Сформулируйте утверждение о том, что любое конечное поле может быть реализовано как факторкольцо кольца многочленов по некоторому идеалу.}
	
	{
		$\;$
		\setlength{\parindent}{0.4cm}
		\hangindent=0.4cm
		
		$\forall$ конечное поле $F_q$, где $q=p^n, \;p$ -- простое, можно реализовать в виде $\mathbb{Z}_p[x]/<h(x)>$, где $h(x)$ -- неприводимый многочлен степени $n$ над $\mathbb{Z}_p$.
		
		$\;$
		\setlength{\parindent}{0cm}
		\hangindent=0cm
	}
	
	\textbf{41. Сформулируйте китайскую теорему об остатках (через изоморфизм колец).}
	
	{
		$\;$
		\setlength{\parindent}{0.4cm}
		\hangindent=0.4cm
		
		Пусть $n\in\mathbb{Z}, n=n_1\cdot\ldots\cdot n_m$, где $n_i$ -- взаимно просты. Тогда кольцо $\mathbb{Z}_n\simeq\mathbb{Z}_{n_1}\times\ldots\times\mathbb{Z}_{n_m}$.
		
		$\;$
		\setlength{\parindent}{0cm}
		\hangindent=0cm
	}
	
	\textbf{42. Сформулируйте утверждение о том, сколько элементов может быть в конечном поле.}
	
		{
		$\;$
		\setlength{\parindent}{0.4cm}
		\hangindent=0.4cm
		
		Число элементов конечного поля всегда $p^n$, где $p$ -- простое, $n\in\mathbb{N}$.
		
		$\;$
		\setlength{\parindent}{0cm}
		\hangindent=0cm
	}
	
	\textbf{43. Дайте определение линейного (векторного) пространства.}
	
	{
		$\;$
		\setlength{\parindent}{0.4cm}
		\hangindent=0.4cm
		
		Пусть $F$ -- поле. Пусть $V$ -- произвольное множество, на котором заданы две операции: сложение и умножение на число. Множество $V$ называется \textit{линейным (векторным) пространством}, если $\forall x, y, z\in V, \forall\lambda\mu\in F$ выполнены следующие 8 свойств:
		
		1) $(x+y)+z=x+(y+z)$ -- ассоциативность сложения
		
		2) $\exists$ нейтральный элемент по сложению: $\exists 0\in V:\forall x\in V\ x+0=0+x=x$
		
		3) $\exists$ противоположный элемент по сложению: $\forall x\in V\;\exists(-x)\in V:x+(-x)=0$
		
		4) $x+y=y+x$ -- коммутативность сложения
		
		5) $\forall x\in V\quad1\cdot x=x$ -- нейтральность $1\in F$
		
		6) ассоциативность умножения на число: $\mu(\lambda x)=(\mu\lambda)x$
		
		7) $(\lambda+\mu)x=\lambda x+\mu x$ -- дистрибутивность относительно умножения на вектор
		
		8) $\lambda(x+y)=\lambda x+\lambda y$ -- дистрибутивность относительно умножения на число
		
		$\;$
		\setlength{\parindent}{0cm}
		\hangindent=0cm
	}
	
	\textbf{44. Дайте определение базиса линейного (векторного) пространства.}
	
	{
		$\;$
		\setlength{\parindent}{0.4cm}
		\hangindent=0.4cm
		
		Базисом линейного пространства $V$ называется система векторов $b_1, \ldots, b_n$, такая, что:
		
		а$\left. \right) $ $b_1, \ldots, b_n$ -- л.н.з.
		
		б$\left. \right) $ любой вектор из $V$ представляется в виде линейной комбинации $b_1, \ldots, b_n$ $\forall x\in V\;\; x=x_1b_1+\ldots+x_nb_n,\; x_i\in F$
		
		$\;$
		\setlength{\parindent}{0cm}
		\hangindent=0cm
	}
	
	\textbf{45. Что такое размерность пространства?}
	
	{
		$\;$
		\setlength{\parindent}{0.4cm}
		\hangindent=0.4cm
		
		Максимальное количество л.н.з. векторов в данном линейном пространстве $V$ называется \textit{размерностью пространства} $V$.
		
		$\;$
		\setlength{\parindent}{0cm}
		\hangindent=0cm
	}
	
	\textbf{46. Дайте определение матрицы перехода от старого базиса линейного пространства к новому.}
	
	{
		$\;$
		\setlength{\parindent}{0.4cm}
		\hangindent=0.4cm
		
		\textit{Матрицей перехода} от базиса $A$ к базису $B$ называется матрица
		\[
		T_{A\rightarrow B}=\begin{pmatrix}
		t_{11}&\ldots&t_{1n}\\
		\vdots&\ddots&\vdots\\
		t_{n1}&\ldots&t_{nn}
		\end{pmatrix}
		\]
		где $t_{1i}, \ldots, t_{ni}$ -- координаты $b_i$ в базисе $A$.
		
		$\;$
		\setlength{\parindent}{0cm}
		\hangindent=0cm
	}
	
	\newpage
	
	\textbf{47. Выпишите формулу для описания изменения координат вектора при изменении базиса.}
	
	{
		$\;$
		\setlength{\parindent}{0.4cm}
		\hangindent=0.4cm
		
		Пусть $x\in V, A$ и $B$ -- базисы в $V$. $x^a=\begin{pmatrix}
		x_1^a\\
		\vdots\\
		x_n^a
		\end{pmatrix}$ -- столбец координат вектора $x$ в базисе $A$, 
		
		$x^b=\begin{pmatrix}
		x_1^b\\
		\vdots\\
		x_n^b
		\end{pmatrix}$ -- столбец координат вектора $x$ в базисе $B$. Тогда: $$x^b=T_{A\rightarrow B}^{-1}\cdot x^a$$
		
		$\;$
		\setlength{\parindent}{0cm}
		\hangindent=0cm
	}
	
	\textbf{48. Дайте определение подпространства в линейном пространстве.}
	
	{
		$\;$
		\setlength{\parindent}{0.4cm}
		\hangindent=0.4cm
		
		Подмножество $W$ векторного пространства $V$ называется \textit{подпространством}, если оно само	является пространством относительно операций в $V$.
		
		$\;$
		\setlength{\parindent}{0cm}
		\hangindent=0cm
	}
	
	\textbf{49. Дайте определения линейной оболочки конечного набора векторов и ранга системы векторов.}
	
	{
		$\;$
		\setlength{\parindent}{0.4cm}
		\hangindent=0.4cm
		
		Множество $L(a_1, \ldots, a_k)=\{\lambda_1a_1+\ldots+\lambda_ka_k|\lambda_i\in F \}$ -- множество всех линейных комбинаций векторов $a_1, \ldots, a_k$ называется \textit{линейной оболочкой} системы $a_1, \ldots a_k$
		
		$\;$
		
		\textit{Рангом} системы векторов $a_1, \ldots, a_k$ в линейном пространстве называется размерность линейной оболочки этой системы $Rg(a_1, \ldots, a_k)=\dim L(a_1, \ldots a_k)$.
		
		$\;$
		\setlength{\parindent}{0cm}
		\hangindent=0cm
	}
	
	\textbf{50. Дайте определения суммы и прямой суммы подпространств.}
	
	{
		$\;$
		\setlength{\parindent}{0.4cm}
		\hangindent=0.4cm
		
		$H_1+H_2=\{x_1+x_2|x_1\in H_1, x_2\in H_2 \}$ называется \textit{суммой} подпространств $H_1$ и $H_2$.
		
		$\;$
		
		$H_1+H_2$ называется \textit{прямой суммой} (и обзначается $H_1\oplus H_2$), если $H_1\cap H_2=\{0\}$, то есть тривиально.
		
		$\;$
		\setlength{\parindent}{0cm}
		\hangindent=0cm
	}
	
	\textbf{51. Сформулируйте утверждение о связи размерности суммы и пересечения подпространств.}
	
	{
		$\;$
		\setlength{\parindent}{0.4cm}
		\hangindent=0.4cm
		
		Пусть $H_1$ и $H_2$ -- подпространства. Тогда $\dim(H_1+H_2)=\dim H_1+\dim H_2-\dim(H_1\cap H_2)$.
		
		$\;$
		\setlength{\parindent}{0cm}
		\hangindent=0cm
	}
	
	\textbf{52. Дайте определение билинейной формы.}
	
	{
		$\;$
		\setlength{\parindent}{0.4cm}
		\hangindent=0.4cm
		
		Функцию $b:V\times V\rightarrow\mathbb{R}$ ($V$ -- линейное пространство над $\mathbb{R}$)  называют \textit{билинейной формой}, если $\forall x, y, z\in V, \;\forall\alpha, \beta\in\mathbb{R}$:
		
		1) $b(\alpha x+\beta y, z)=\alpha b(x, z)+\beta b(y, z)$
		
		2) $b(x, \alpha y+\beta z)=\alpha b(x, y)+\beta b(x, z)$
		
		$\;$
		\setlength{\parindent}{0cm}
		\hangindent=0cm
	}
	
	\textbf{53. Дайте определение квадратичной формы.}
	
	{
		$\;$
		\setlength{\parindent}{0.4cm}
		\hangindent=0.4cm
		
		Однородный многочлен второй степени от $n$ переменных, то есть: $$Q(x)=\sum\limits_{i=1}^n a_{ii}x_i^2+2\sum\limits_{1\leq i<j\leq n} a_{ij}x_ix_j,\; a_{ij}\in\mathbb{R}$$называют \textit{квадратичной формой}.
		
		$\;$
		\setlength{\parindent}{0cm}
		\hangindent=0cm
	}
	
	\textbf{54. Дайте определения положительной и отрицательной определенности квадратичной формы.}
	
	{
		$\;$
		\setlength{\parindent}{0.4cm}
		\hangindent=0.4cm
		
		Квадратичную форму $Q(x)$ называют:
		
		\begin{itemize}
		\item \textit{положительно определенной}, если $\forall x\ne 0\ Q(x)>0$
		\item \textit{отрицательно определенной}, если $\forall x\ne 0\ Q(x)<0$\\
		\end{itemize}	
		
		$\;$
		\setlength{\parindent}{0cm}
		\hangindent=0cm
	}
	
	\textbf{55. Какую квадратичную форму называют знакопеременной?}
	
	{
		$\;$
		\setlength{\parindent}{0.4cm}
		\hangindent=0.4cm
		
		Квадратичную форму $Q(x)$ называют \textit{знакопеременной}, если $\exists x, y\in V\ Q(y)<0<Q(x)$.
		
		$\;$
		\setlength{\parindent}{0cm}
		\hangindent=0cm
	}
	
	\textbf{56. Дайте определения канонического и нормального вида квадратичной формы.}
	
	{
		$\;$
		\setlength{\parindent}{0.4cm}
		\hangindent=0.4cm
		
		Квадратичную форму $Q(x)=\alpha_1x_1^2+\ldots+\alpha_nx_n^2, \; \alpha_i\in\mathbb{R}\ i=\overline{1, n}$ (то есть не имеющую попарных произведений переменных) называют квадратичной формой \textit{канонического вида}.
		
		$\;$
		
		Если $\alpha_i\in\{1, -1, 0\}$, то канонический вид называется \textit{нормальным}.
		
		$\;$
		\setlength{\parindent}{0cm}
		\hangindent=0cm
	}
	
	\textbf{57. Как меняется матрица билинейной формы при замене базиса? Как меняется матрица квадратичной формы при замене базиса?}
	
	{
		$\;$
		\setlength{\parindent}{0.4cm}
		\hangindent=0.4cm
		
		Пусть $U$ -- матрица перехода от базиса $e$ к базису $f$. Пусть $B_e$ -- матрица билинейной формы в базисе $e$, $B_f$ -- матрица билинейной формы в базисе $f$. Тогда: $$B_f=U^TB_eU$$
		
		При переходе от базиса $e$ к базису $e'$ линейного пространства $V$ матрица квадратичной формы меняется следующим образом: $$A'=S^TAS$$ где $S$ -- матрица перехода от $e$ к $e'$.
		
		$\;$
		\setlength{\parindent}{0cm}
		\hangindent=0cm
	}
	
	\textbf{58. Сформулируйте критерий Сильвестра и его следствие.}
	
	{
		$\;$
		\setlength{\parindent}{0.4cm}
		\hangindent=0.4cm
		
		Квадратичная форма $Q(x)$ от $n$ переменных $x=(x_1, \ldots, x_n)^T$ положительно определена $\Leftrightarrow\begin{cases}
		\triangle_1>0\\
		\vdots\\
		\triangle_n>0
		\end{cases}$. Здесь $Q(x)=x^TAx$, $$A=\begin{pmatrix}
		a_{11}&a_{12}&\ldots&a_{1n}\\
		a_{21}&a_{22}& & \vdots\\
		\vdots& &\ddots&\vdots \\
		a_{n1}&\ldots &\ldots&a_{nn}
		\end{pmatrix},	\;\triangle_1=a_{11},\; \triangle_2=\begin{vmatrix}
		a_{11}&a_{12}\\
		a_{21}&a_{22}
		\end{vmatrix}, \ldots,\; \triangle_n=\det A$$
		
		то есть последовательность главных угловых миноров положительна.
		
		$\;$
		
		\textbf{Следствие:} $Q(x)$ отрицательно определена $\Leftrightarrow\triangle_1<0, \triangle_2>0, \ldots, (-1)^n\triangle_n>0$ (Знаки главных угловых миноров чередуются, начиная с минуса).
		
		$\;$
		\setlength{\parindent}{0cm}
		\hangindent=0cm
	}
	
	\textbf{59. Сформулируйте закон инерции квадратичных форм. Что такое индексы инерции?}
	
	{
		$\;$
		\setlength{\parindent}{0.4cm}
		\hangindent=0.4cm
		
		Для любых двух канонических видов одной и той квадратичной формы
		$$Q_1(y_1, \ldots, y_m)=\lambda_1y_1^2+\ldots+\lambda_my_m^2, \lambda_i\ne0, i=\overline{1, m}$$
		$$Q_2(z_1, \ldots, z_k)=\mu_1z_1^2+\ldots+\mu_kz_k^2, \mu_j\ne0, j=\overline{1, k}$$
		
		$1)\; m=k=RgA$ -- рангу квадратичной формы
		
		2) количество положительных $\lambda_i=$ количеству положительных $\mu_j=i_+$ -- \textit{положительный индекс инерции}. Количество отрицательных $\lambda_i=$ количеству отрицательных $\mu_j=i_-$ -- \textit{отрицательный индекс инерции}.
		
		$\;$
		\setlength{\parindent}{0cm}
		\hangindent=0cm
	}
	
	\newpage
	
	\textbf{60. Дайте определение линейного отображения. Приведите пример.}
	
	{
		$\;$
		\setlength{\parindent}{0.4cm}
		\hangindent=0.4cm
		
		Отображение $\varphi: V_1\rightarrow V_2$ называется \textit{линейным}, если:
		
		1) $\forall u, v\in V_1, \ \varphi(u+v)=\varphi(u)+\varphi(v)$
		
		2) $\forall u\in V_1, \forall \lambda\in F\ \varphi(\lambda u)=\lambda\varphi(u)$
		
		$\;$
		
		\textbf{Пример:} В линейном пространстве $m\times n$ матриц существует линейное отображение умножения слева на фиксированную матрицу $A_{l\times m}: \varphi:X\rightarrow A\cdot X$.
		
		$\;$
		\setlength{\parindent}{0cm}
		\hangindent=0cm
	}
	
	\textbf{61. Дайте определение матрицы линейного отображения.}
	
	{
		$\;$
		\setlength{\parindent}{0.4cm}
		\hangindent=0.4cm
		
		\textit{Матрица линейного отображения} -- это матрица $A=\begin{pmatrix}
		a_{11}&\ldots&a_{1n}\\
		\vdots&\ddots&\vdots\\
		a_{m1}&\ldots&a_{mn}
		\end{pmatrix}$, где по столбцам стоят координаты образов векторов базиса $V_1$ в базисе $V_2$.
		
		$\;$
		\setlength{\parindent}{0cm}
		\hangindent=0cm
	}
	
	\textbf{62. Выпишите формулу преобразования матрицы линейного отображения при замене базиса. Как выглядит формула в случае линейного оператора?}
	
	{
		$\;$
		\setlength{\parindent}{0.4cm}
		\hangindent=0.4cm
		
		Пусть $\varphi$ -- линейное отображение из линейного пространства $V_1$ в линейное пространство $V_2$. Пусть $A_{e_1e_2}$ -- матрица линейного отображения в паре базисов: $e_1$  в пространстве $V_1$ и $e_2$ в пространстве $V_2$ и пусть $T_1$ -- матрица перехода от $e_1$ к $e_1'$, $T_2$ -- матрица перехода от $e_2$ к $e_2'$. Тогда: $$A_{e_1'e_2'}=T_2^{-1}A_{e_1e_2}T_1$$. 
		
		Формула для линейных операторов: $$A_{E'}=T^{-1}A_ET$$
		
		$\;$
		\setlength{\parindent}{0cm}
		\hangindent=0cm
	}
	
	\Large
	\centering
	
	$\;$
	
	\textbf{4-й модуль}
	
	\flushleft
	\small
	
	\textbf{1. Дайте определения собственного вектора и собственного значения линейного оператора.}
	
	$\;$
	{
		\setlength{\parindent}{0.4cm}
		\hangindent=0.4cm
		
		Число $\lambda$ называется \textit{собственным числом} или \textit{собственным значением} линейного оператора $A:V\rightarrow V$, если существует вектор $v\in V, v \not = 0$, такой, что $Av=\lambda v$. При этом вектор $v$ называется \textit{собственным вектором}, отвечающим за собственное значение $\lambda$.
		
		$\;$
		\setlength{\parindent}{0cm}
		\hangindent=0cm
	}
	
	\textbf{2. Дайте определения характеристического уравнения и характеристического многочлена квадратной матрицы.}
	
	$\;$
	{
		\setlength{\parindent}{0.4cm}
		\hangindent=0.4cm
		
		Для произвольной квадратной матрицы $A$ определитель $\chi_A(\lambda) = \det(A-\lambda E)$ называют \textit{характеристическим многочленом} матрицы $A$. \textit{Характеристическое уравнение} - уравнение вида $\det(A-\lambda E) = 0.$ 
		
		$\;$
		\setlength{\parindent}{0cm}
		\hangindent=0cm
	}
	
	\textbf{3. Сформулируйте утверждение о связи характеристического уравнения и спектра линейного оператора.}
	
	$\;$
	{
		\setlength{\parindent}{0.4cm}
		\hangindent=0.4cm
		
		$\lambda$ принадлежит спектру линейного оператора $\Leftrightarrow$ $\lambda$ - корень характеристического уравнения(над алгебраически замкнутым полем).
		
		$\;$
		\setlength{\parindent}{0cm}
		\hangindent=0cm
	}
	
	\textbf{4. Дайте определение собственного подпространства.}
	
	$\;$
	{
		\setlength{\parindent}{0.4cm}
		\hangindent=0.4cm
		
		Пусть $A:V\rightarrow V$ - линейный оператор, $\lambda$ - собственное значение $A$. Тогда множество $V_\lambda = \{v\in V| Av=\lambda v\}$ - подпространство в $V$, называемое \textit{собственным подпространством}, отвечающим $\lambda$.
		
		$\;$
		\setlength{\parindent}{0cm}
		\hangindent=0cm
	}
	
	\textbf{5. Дайте определения алгебраической и геометрической кратности собственного значения. Какое неравенство их связывает?}
	
	$\;$
	{
		\setlength{\parindent}{0.4cm}
		\hangindent=0.4cm
		
		\textit{Алгебраической кратностью} $\lambda$ называетсяя кратность $\lambda$ как корня характеристического уравнения. Размерность подпространтсва $V_\lambda$ называется \textit{геометрической кратностью}
		собственного значения $\lambda$. Геометрическая кратность собственного значения не превышает его алгебраической кратности.
		
		$\;$
		\setlength{\parindent}{0cm}
		\hangindent=0cm
	}
	
	\textbf{6. Дайте определение следа матрицы. Как меняется след матрицы оператора при замене базиса.}
	
	$\;$
	{
		\setlength{\parindent}{0.4cm}
		\hangindent=0.4cm
		
		\textit{Следом} матрицы $A$ называется сумма ее диагональных элементов: $tr A = \sum a_ { ii }$. След матрицы не зависит от выбора базиса.
		
		$\;$
		\setlength{\parindent}{0cm}
		\hangindent=0cm
	}
	
	\textbf{7. Каким свойством обладают собственные векторы линейного оператора, отвечающие различным собственным значениям.}
	
	$\;$
	{
		\setlength{\parindent}{0.4cm}
		\hangindent=0.4cm
		
		Пусть $\lambda_1,\dots,\lambda_k$ - собственные значения линейного оператора $A$, $\lambda_i \not = \lambda_j$, а $v_1, \dots, v_k$ - соответствующие собственные векторы. Тогда $v_1, \dots, v_k$ - линейно независимые, т.е. собственные векторы, отвечающие различным собственным значениям, линейно независимы.
		
		$\;$
		\setlength{\parindent}{0cm}
		\hangindent=0cm
	}
	
	\textbf{8. Сформулируйте критерий диагональности матрицы оператора.}
	
	$\;$
	{
		\setlength{\parindent}{0.4cm}
		\hangindent=0.4cm	
		
		Матрица линейного оператора является диагональной в этом базисе $\Leftrightarrow$ все векторы этого базиса являются собственными векторами для $A$.
		
		$\;$
		\setlength{\parindent}{0cm}
		\hangindent=0cm
	}
	
	\textbf{9. Сформулируйте критерий диагонализируемости матрицы оператора с использованием понятия геометрической кратности.}
	
	$\;$
	{
		\setlength{\parindent}{0.4cm}
		\hangindent=0.4cm	
		
		Матрицы линейного оператора приводится к диагональному виду $\Leftrightarrow$ геометрическая кратность каждого собственного значения орператора равна его алгебраической кратности
		
		$\;$
		\setlength{\parindent}{0cm}
		\hangindent=0cm
	}
	
	\textbf{10. Дайте определение жордановой клетки. Сформулируйте теорему о жордановой нормальной форме матрицы оператора.}
	
	$\;$
	{
		\setlength{\parindent}{0.4cm}
		\hangindent=0.4cm	
		
		\textit{Жорданова клетка} размера $m \times m$ - это матрица вида: $$
		J_m(\lambda_i)=\begin{pmatrix}
		\lambda_i & 1 & \dots & 0 \\
		& \ddots & \ddots & \vdots \\
		&  & \lambda_{i} & 1 \\
		0&  &  & \lambda_i 
		\end{pmatrix}
		$$
		
		$\forall A \in Mn(\mathbb{F})$ приводится заменой базиса к ЖНФ над алгебраически замкнутым полем (например $\mathbb{C}$). Иными словами $\exists C \in Mn(\mathbb{F})$ и $\det C \not = 0$, что $A = C J C^{-1}$, где $J$ - ЖНФ.
		
		$\;$
		\setlength{\parindent}{0cm}
		\hangindent=0cm
	}
	
	\textbf{11. Выпишите формулу для количества жордановых клеток заданного размера.}
	
	$\;$
	{
		\setlength{\parindent}{0.4cm}
		\hangindent=0.4cm	
		
		$h_k(\lambda_i)=\rho_{k+1}-2\rho_{k}+\rho_{k-1}$  - количество жордановых клеток с $\lambda_i$ на диагонали размера $k\times k$ ($\rho_j = Rg(A-\lambda_i E)^j, \; \rho_0 = n$).
		
		$\;$
		\setlength{\parindent}{0cm}
		\hangindent=0cm
	}
	
	\textbf{12. Сформулируйте теорему Гамильтона-Кэли.}
	
	$\;$
	{
		\setlength{\parindent}{0.4cm}
		\hangindent=0.4cm	
		
		Если $A$ - квадратная матрица и $\chi(\lambda)$ её характеристический многочлен, то $\chi(A) = 0.$
		
		$\;$
		\setlength{\parindent}{0cm}
		\hangindent=0cm
	}
	
	\textbf {13. Дайте определение корневого подпространства.}
	
	{
		\setlength{\parindent}{0.4cm}
		\hangindent=0.4cm	
		$\;$
		
		Корневое подпространство: $K_i = Ker(A-\lambda_{i}E)^{m_i}$, где $m_i$ - алгебраическая кратность $\lambda_i$.
		
		$\;$
		\setlength{\parindent}{0cm}
		\hangindent=0cm
	}
	
	\textbf{14.	Дайте определение минимального многочлена линейного оператора.}
	
	$\;$
	{
		\setlength{\parindent}{0.4cm}
		\hangindent=0.4cm	
		
		Для матрицы $A$ многочлен $\mu(x)$ называется \textit{минимальным}, если $\mu(A) = 0$ и $\forall f : f(A) = 0, \; \deg(f) \geq \deg(\mu)$.
		
		$\;$
		\setlength{\parindent}{0cm}
		\hangindent=0cm
	}
	
	\textbf{15. Дайте определение инвариантного подпространства.}
	
	$\;$
	{
		\setlength{\parindent}{0.4cm}
		\hangindent=0.4cm	
		
		Подпространство $L$ векторного пространства $V$ называется \textit{инвариантным} относительно оператора $\varphi$, если $\varphi(x)\in L \; \forall x \in L.$
		
		$\;$
		\setlength{\parindent}{0cm}
		\hangindent=0cm
	}
	
	\textbf{16. Дайте определение евклидова пространства.}
	
	$\;$
	{
		\setlength{\parindent}{0.4cm}
		\hangindent=0.4cm	
		
		\textit{Евклидово пространство} - это пара $V$ - линейное пространство над $\mathbb{R}$ и скалярное произведение $g(x,y)$, то есть симметричная положительно определенная билинейная форма.
		
		$\;$
		
		\setlength{\parindent}{0.8cm}
		\hangindent=0.8cm
		
		$\mathbb{E} = (V, g(x,y))$ и $\forall x, y \in V, \forall \lambda \in \mathbb{R}$:
		
		\begin{itemize}
			
			\item  $g(x,y) = g(y,x)$
			\item  $g(x + y, z) = g(x, z) + g(y, z)$
			\item $g(\lambda x, y) = \lambda g(x, y)$
			\item  $g(x, x) \geq 0$ и $ g(x, x) = 0 \Leftrightarrow x= 0$
			
		\end{itemize}
		
		\setlength{\parindent}{0.4cm}
		\hangindent=0.4cm
		
		$\;$
		\setlength{\parindent}{0cm}
		\hangindent=0cm
	}
	
	\textbf{17. Выпишите неравенство Коши-Буняковского и треугольника.}
	
	$\;$
	{
		\setlength{\parindent}{0.4cm}
		\hangindent=0.4cm
		
		Неравенсво Коши-Буняковского: $\forall x, y \in \mathbb{E} \  |(x, y)|\leq||x||\cdot||y||. $
		
		Неравенсво треугольника: $\forall x, y \in \mathbb{E} \ \Vert x + y \Vert \leq \Vert x \Vert + \Vert y \Vert. $
		
		$\;$
		\setlength{\parindent}{0cm}
		\hangindent=0cm
	}
	
	\textbf{18. Дайте определения ортогонального и ортонормированного базисов.}
	
	$\;$
	{
		\setlength{\parindent}{0.4cm}
		\hangindent=0.4cm
		
		Пусть $\{v_1, \ldots, v_k \}$ -- ортогональная система векторов, причем $v_i\ne0\ \forall i=\overline{1, k}$. 
       
       Если $k=\dim V=n$, то $v_1, \ldots, v_k$ будут \textit{ортогональным} базисом
       
       $\;$
       
       Если рассмотрим $e_1=\dfrac{v_1}{||v_1||}, \ldots, e_n=\dfrac{v_n}{||v_n||}$, то мы получим \textit{ОНБ}\\
		
		$\;$
		\setlength{\parindent}{0cm}
		\hangindent=0cm
	}
	
	\textbf{19. Опишите алгоритм ортогонализации Грама-Шмидта.}
	
	$\;$
	{
		\setlength{\parindent}{0.4cm}
		\hangindent=0.4cm
		
		Пусть имеется система линейно независимых векторов ($a_1, \dots, a_n).$ Определим оператор проекции следующим образом: $proj_ba= \frac{(a, b)}{(b,b)}b.$ Этот оператор проецирует вектор $a$ коллинеарно вектору $b$. 
		
		$\;$
		
		Классический процесс Грама — Шмидта выполняется следующим образом:
		
		$\;$
		
		\setlength{\parindent}{0.8cm}
		\hangindent=0.8cm
		
		$b_1=a_1$
		
		$b_2=a_2 - proj_{b_1}a_2$
		
		$\vdots$
		
		$b_n = a_n - \sum_{j=1}^{n-1} proj_{b_j}a_n$
		
		\setlength{\parindent}{0.4cm}
		\hangindent=0.4cm
		
		$\;$
		
		В результате получим систему ортогональных векторов $(b_1, \dots, b_n).$
		
		$\;$
		\setlength{\parindent}{0cm}
		\hangindent=0cm
	}
	
	\textbf{20. Дайте определение матрицы Грама.}
	
	$\;$
	{
		\setlength{\parindent}{0.4cm}
		\hangindent=0.4cm
		
		\textit{Матрицей Грама} системы векторов $(e_1, \dots, e_n)$ называется квадратная матрица, состоящая из всевозможных скалярных произведений этих векторов:
		
		$$
		\Gamma = 
		\begin{pmatrix}
		(e_1, e_1) & (e_1, e_2) &\dotsb& (e_1, e_n) \\
		(e_2, e_1) & (e_2, e_2) & \dotsb & (e_2, e_n) \\
		\vdots &&& \\
		(e_n, e_1) & (e_n, e_2) & \dotsb & (e_n, e_n) 
		\end{pmatrix}
		$$	
		
		$\;$
		\setlength{\parindent}{0cm}
		\hangindent=0cm
	}
	
	\textbf{21. Выпишите формулу для преобразования матрицы Грама при переходе к новому базису.}
	
	$\;$
	{
		\setlength{\parindent}{0.4cm}
		\hangindent=0.4cm
		
		Матрицы Грама двух базисов $e$ и $e'$ связаны соотношением $\Gamma' = U^T \Gamma U $, где $U$ - матрица перехода от $e$ к $e'$.
		
		$\;$
		\setlength{\parindent}{0cm}
		\hangindent=0cm
	}

	\textbf{22. Сформулируйте критерий линейной зависимости с помощью матрицы Грама.}
	
	$\;$
	{
		\setlength{\parindent}{0.4cm}
		\hangindent=0.4cm
		
		Система векторов $e_1, \dots, e_n$
		линейно зависима $\Leftrightarrow$ определитель матрицы Грама этой системы равен нулю.
		
		$\;$
		\setlength{\parindent}{0cm}
		\hangindent=0cm
	}


	\textbf{23. Дайте определение ортогонального дополнения.}
	
	$\;$
	{
		\setlength{\parindent}{0.4cm}
		\hangindent=0.4cm
		
		Пусть $H \subseteq V$. Множество $H^\perp=\{x\in V | (x,y)=0 \; \forall y \in H\}$ называется \textit{ортогональным дополнением}.
		
		$\;$
		\setlength{\parindent}{0cm}
		\hangindent=0cm
	}
	
	\textbf{24. Дайте определения ортогональной проекции вектора на подпространство и ортогональной составляющей.}
	
	{
		$\;$
		\setlength{\parindent}{0.4cm}
		\hangindent=0.4cm
		
		Пусть $L$ - линейное подпространство евклидова пространства
		$\mathbb{E}$, $a$ - произвольный вектор пространства $\mathbb{E}$. Если $a = b + c$, причём $b \in L, c \in L ^ \perp$, то $b$ называется \textit{ортогональной проекцией} вектора $a$ на
		подпространство $L$ $(proj_La)$, а $c$ - \textit{ортогональной составляющей} при (ортогональном) проектировании вектора a на подпространство $(ort_La)$.
		
		$\;$
		\setlength{\parindent}{0cm}
		\hangindent=0cm
	}
	
	\textbf{25. Выпишите формулу для ортогональной проекции вектора на подпространство, заданное как линейная оболочка данного линейно независимого набора векторов.}
	
	{
		$\;$
		\setlength{\parindent}{0.4cm}
		\hangindent=0.4cm
		
		Пусть $L=\langle a_1, \dots, a_n \rangle.$ Тогда $proj_L x=A(A^T A)^{-1}A^Tx$, где $A$ - матрица, составленная из столбцов $a_1, \dots, a_n$.
			
		$\;$
		\setlength{\parindent}{0cm}
		\hangindent=0cm
	}

	\textbf{26. Выпишите формулу для вычисления расстояния с помощью определителей матриц Грама.}
	
	{
		$\;$
		\setlength{\parindent}{0.4cm}
		\hangindent=0.4cm
		
		Пусть $S \subset \mathbb{E}$ - подпространство, $x \in \mathbb{E}, (e_1, \dots, e_n)$ - базис $S$. Тогда:
		
		$$
		(p(x, S))^2 = \frac{\det G(e_1,\dots, e_n, x)}{\det G(e_1, \dots, e_n)}
		$$
		
		$\;$
		\setlength{\parindent}{0cm}
		\hangindent=0cm
	}
	
	\textbf{27. Дайте определение сопряженного пространства.}
	
	{
		$\;$
		\setlength{\parindent}{0.4cm}
		\hangindent=0.4cm
		
		Пространством сопряженным к линейному пространству $L$ называется множетсво всех линейных форм на нем с операциями:
		$$
		\forall x \in L \; (f_1 + f_2)(x) = f_1(x) + f_2(x)
		$$
		$$
	 	\forall \lambda \in \mathbb{F} \; (\lambda f)(x) = \lambda f(x)
		$$
		
		Обозначение: $L^* \subseteq Hom(L, \mathbb{F})$.
		
		$\;$
		\setlength{\parindent}{0cm}
		\hangindent=0cm
	}

	\textbf{28. Выпишите формулу для преобразования координат ковектора при переходе к другому базису.}
	
	{
		$\;$
		\setlength{\parindent}{0.4cm}
		\hangindent=0.4cm
		
		Пусть $L^*$ - сопряженное пространство. Если записывать координаты элементов по столбцам, то при переходе к другому базису они будут преобразовываться по формуле:
		$$
		[f]_g^{\text{ст}} = T_{e \rightarrow g}^T \cdot [f]_e^{\text{ст}}
		$$
		
		$\;$
		\setlength{\parindent}{0cm}
		\hangindent=0cm
	}

	\textbf{29. Дайте определение взаимных базисов.}
	
	{
		$\;$
		\setlength{\parindent}{0.4cm}
		\hangindent=0.4cm
		
		Базис $\mathfrak{e}=(e_1,\dots,e_n)$ в линейном пространстве $L$ и базис $\mathfrak{f} = (f_1, \dots,f_n)$ в сопряженном пространстве $L^*$ называют \textit{взаимными}, если:
		$$
		(e_i, f^j) = \delta_i^j=\begin{cases}
		1, \; i = j \\
		0, \; i \not = j
		\end{cases}
		$$
		
		$\;$
		\setlength{\parindent}{0cm}
		\hangindent=0cm
	}
	\textbf{}\\
	\textbf{}\\

	\textbf{30. Дайте определение биортогонального базиса.}
	
	{
		$\;$
		\setlength{\parindent}{0.4cm}
		\hangindent=0.4cm
		
		Если $L = L^*$, то взаимный к данному базис называется \textit{биортогональным}.
		
		$\;$
		\setlength{\parindent}{0cm}
		\hangindent=0cm
	}

	\textbf{31. Дайте определение сопряженного оператора в евклидовом пространстве.}
	
	{
		$\;$
		\setlength{\parindent}{0.4cm}
		\hangindent=0.4cm
		
		Линейный оператор $\mathcal{A}^*$ называется \textit{сопряженным} к линейному оператору $\mathcal{A}$, если $\forall x, y \in \mathbb{E}$ верно, что $(\mathcal{A}x, y)= (x, \mathcal{A}^*y).$
		
		$\;$
		\setlength{\parindent}{0cm}
		\hangindent=0cm
	}

	\textbf{32. Дайте определение самосопряженного (симметрического) оператора.}
	
	{
		$\;$
		\setlength{\parindent}{0.4cm}
		\hangindent=0.4cm
		
		Линейный оператор $\mathcal{A}$ называется \textit{самосопряженным (симметричным) }, если $\forall x, y \in \mathbb{E}$ верно, что $(\mathcal{A}x, y)= (x, \mathcal{A}y)$, т.е. $\mathcal{A}^*=\mathcal{A}$.
		
		$\;$
		\setlength{\parindent}{0cm}
		\hangindent=0cm
	}

	\textbf{33. Дайте определение ортогонального оператора.}
	
	{
		$\;$
		\setlength{\parindent}{0.4cm}
		\hangindent=0.4cm
		
		Линейный оператор $\mathcal{A}$ называется \textit{ортогональным}, если $\forall x, y \in \mathbb{E}$ верно, что $(\mathcal{A}x, \mathcal{A}y)= (x, y)$, т.е. оператор сохраняет скалярное произведение, и значит, он сохраняет длины сторон и углы между ними.
		
		$\;$
		\setlength{\parindent}{0cm}
		\hangindent=0cm
	}


	\textbf{34. Как найти матрицу сопряженного оператора в произвольном базисе?}
	
	{
		$\;$
		\setlength{\parindent}{0.4cm}
		\hangindent=0.4cm
		
		Пусть $\mathfrak{e} = (e_1,\dots,e_n)$ - базис в $\mathbb{E}$, $\Gamma$ - матрица Грама, $\mathcal{A}$ - матрица линейного оператора. Тогда матрица сопряженного линейного оператора выражается как:
		$$
		\mathcal{A}^*=\Gamma^{-1}A^T \Gamma
		$$
		$\;$
		\setlength{\parindent}{0cm}
		\hangindent=0cm
	}

	\textbf{35. Каким свойством обладают собственные значения самосопряженного оператора?}
	
	{
		$\;$
		\setlength{\parindent}{0.4cm}
		\hangindent=0.4cm
		
		Собственные значения самосопряженного оператора вещественны.
		
		$\;$
		\setlength{\parindent}{0cm}
		\hangindent=0cm
	}

	\textbf{36. Что можно сказать про собственные векторы самосопряженного оператора, отвечающие разным собственным значениям?}
	
	{
		$\;$
		
		\setlength{\parindent}{0.4cm}
		\hangindent=0.4cm
		
		Собственные векторы, принадлежащие различным собственным значениям самосопряженного преобразования, ортогональны.
		
		$\;$
		\setlength{\parindent}{0cm}
		\hangindent=0cm
	}

	\textbf{37. Сформулируйте определение ортогональной матрицы.}
	
	{
		$\;$
		\setlength{\parindent}{0.4cm}
		\hangindent=0.4cm
		
		Матрица $C \in Mat_n(\mathbb{R})$ называется \textit{ортогональной}, если $C^T C = E$.
		
		$\;$
		\setlength{\parindent}{0cm}
		\hangindent=0cm
	}

	\textbf{38. Сформулируйте критерий ортогональности оператора, использующий его матрицу.}
	
	{
		$\;$
		\setlength{\parindent}{0.4cm}
		\hangindent=0.4cm
		
		Матрица линейного оператора $\mathcal{A}$ в ОНБ ортогональна $\Leftrightarrow$ $\mathcal{A}$ - ортогональный оператор.
		$\;$
		\setlength{\parindent}{0cm}
		\hangindent=0cm
	}

	\newpage

	\textbf{39. Каков канонический вид ортогонального оператора? Сформулируйте теорему Эйлера.}
	
	{
		$\;$
		\setlength{\parindent}{0.4cm}
		\hangindent=0.4cm
		
		Для любого отогонального оператора $\mathcal{A}$ существует ортонормированный базис, в котором матрица оператора имеет следующий вид:
		
		\begin{center}
		
		$
		\mathcal{A} = 
		\begin{pmatrix}
		\Pi(\alpha_1) & && &&&&& \\
		& \ddots &&&&&& &\\
		&& \Pi(\alpha_k)&&&&&& \\
		&&& -1 &&&& &\\
		&&&& \ddots &&&& \\
		&&&&& -1 &&& \\
		&&&&&& 1 && \\
		&&&&&&& \ddots & \\
		&&&&&&&& 1 
		\end{pmatrix}
		$ , где $\Pi(\alpha) = \begin{pmatrix}
		\cos \alpha & -\sin \alpha \\
		\sin \alpha & \cos \alpha
		\end{pmatrix}$
		
		\end{center}
		
		$\;$
		
		\textbf{Теорема Эйлера.} \\
		$\forall$ ортогонального преобразования в $\mathbb{R}^3\ \exists$ ОНБ, в котором его матрица имеет вид$:$
		
        $A=\begin{pmatrix}
        \cos\varphi&-\sin\varphi&0\\
        \sin\varphi&\cos\varphi&0\\
        0&0&\pm1
        \end{pmatrix}$.\\
		
		$\:$ 
		
		\setlength{\parindent}{0cm}
		\hangindent=0cm
	}

	\textbf{40. Сформулируйте теорему о существовании для самосопряженного оператора базиса из собственных векторов.}
	
	{
		$\;$
		\setlength{\parindent}{0.4cm}
		\hangindent=0.4cm
		
		Для всякого самосопряженного оператора $\mathcal{A}$ существует ортонормированный базис из собственных векторов, в котором матрица оператора имеет диагональный вид.
		$$\Lambda=diag(\lambda_1,\dots,\lambda_n)$$
		
		
		$\lambda_1,\dots,\lambda_n$ - собственные значения оператора $\mathcal{A}$, повторенные в соответствии с их кратностью.
		
		$\;$
		\setlength{\parindent}{0cm}
		\hangindent=0cm
	}


	\textbf{41. Сформулируйте теорему о сингулярном разложении.}
	
	{
		$\;$
		\setlength{\parindent}{0.4cm}
		\hangindent=0.4cm
		
		Для любой матрицы $A \in Mat_{m \times n}(\mathbb{R})$ существуют ортогональные матрицы $V \in M_m(\mathbb{R})$ и $W \in M_n(\mathbb{R})$ и диагональная матрица $\Sigma \in Mat_{m \times n}(\mathbb{R})$, такие что:
		
		\begin{center}
		$
		A = V\Sigma W^T
		$, где 
		$\Sigma = \left(
		\begin{array}{ccc|ccc}
		\sigma_1 &&&&& \\
		& \ddots &&&0& \\
		&& \sigma_r &&& \\
		\hline	
		&&&0&& \\
		&0&&&\ddots & \\
		&&&&&0
		\end{array} \right)
		$ , $\sigma_1 \ge \sigma_2 \ge \ldots \ge \sigma_r > 0$
		
		\end{center}
		
		$\;$
		\setlength{\parindent}{0cm}
		\hangindent=0cm
	}

	\textbf{42. Сформулируйте утверждение о QR-разложении.}
	
	{
		$\;$
		\setlength{\parindent}{0.4cm}
		\hangindent=0.4cm
		
		Пусть $A\in M_m(\mathbb{R})$ и столбцы $A_1, \ldots, A_m$ л.н.з. Тогда $\exists\ Q$ и $R:A=QR$, причем $Q$ -- ортогональная матрица, 
		
		$R$ -- верхнетреугольная матрица\\
		
		$\;$
		\setlength{\parindent}{0cm}
		\hangindent=0cm
	}

	\textbf{43. Сформулируйте утверждение о полярном разложении.}
	
	{
		$\;$
		\setlength{\parindent}{0.4cm}
		\hangindent=0.4cm
		
		$\forall$ матрица $A\in M_n(\mathbb{R})$ представима в виде $A=SU$, где $S$ -- симметрическая матрица с положительными собственными значениями, а $U$ -- ортогональная.\\
		
		$\;$
		\setlength{\parindent}{0cm}
		\hangindent=0cm
	}

	\textbf{44. Что можно сказать про ортогональное дополнение к образу сопряженного оператора?}
	
	{
		$\;$
		\setlength{\parindent}{0.4cm}
		\hangindent=0.4cm
		
		Пусть линейный оператор $A:E\rightarrow E$. Тогда $E=KerA\oplus ImA^*$\\
		
		$\;$
		\setlength{\parindent}{0cm}
		\hangindent=0cm
	}

	\textbf{45. Сформулируйте теорему Фредгольма и альтернативу Фредгольма.}
	
	{
		$\;$
		\setlength{\parindent}{0.4cm}
		\hangindent=0.4cm
		
		\textit{Теорема Фредгольма}\\
       $Ax=b$ совместна$\Leftrightarrow$вектор $b\perp$всем решениям однородной СЛАУ $A^Ty=0$\\
       \textit{Альтернатива Фредгольма}\\
       Либо у $Ax=b\ \exists!$ решение $\forall b$, либо $A^Ty=0$ имеет ненулевое решение\\
		
		$\;$
		\setlength{\parindent}{0cm}
		\hangindent=0cm
	}


	
\end{document}	
